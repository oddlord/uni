\section*{Esercizio 9.6}

    \enunciato{1}{
        Si aggiunga al linguaggio SMALL un comando \textbf{stop} con la semantica informale di far terminare il programma. Se ne dia la semantica e si dimostri che $c_1;\mbox{\textbf{stop}}$ e $c_1;\mbox{\textbf{stop}};c_2$ sono semanticamente equivalenti.
    }
    
    Banalmente, essendo \textbf{stop} un nuovo comando, per aggiornare la semantica di SMALL basta aggiungere una nuova clausola per i comandi contenente \textbf{stop}, quindi si avrà la sintassi dei comandi come segue:
    \begin{align*}
        c \quad ::= \quad & e:=e_1 \orbar c_1;c_2 \orbar \mbox{\textbf{if}}\ e \ \mbox{\textbf{then}}\ c_1\ \mbox{\textbf{else}}\ c_2 \orbar \mbox{\textbf{while}}\ e\ \mbox{\textbf{do}}\ c \orbar \mbox{\textbf{output}}\ e \\
        &\orbar \mbox{\textbf{begin}}\ d;c\ \mbox{\textbf{end}}\ \orbar e(e_1) \orbar \mbox{\textbf{stop}}
    \end{align*}
    
    Ci possono essere varie soluzioni per definire la semantica del nuovo comando \textbf{stop}. La più semplice è quella di far produrre al comando \textbf{stop} un nuovo elemento, che chiameremo \textit{stop}, e trattarlo im modo simile all'elemento \textit{error}. Infatti se ci pensiamo bene, un errore fa quello che vorrebbe fare il comando \textbf{stop}, ovvero terminare il programma senza eseguire altri comandi. L'unica differenza è che, mentre \textit{error} propaga l'errore anche come output del programma stesso, utilizzando \textbf{stop} ci aspettiamo invece di fornire comunque un qualche tipo di output, in particolare il contenuto della memoria nella locazione riservata $lout$.\\
    \\
    Per far questo, ridefiniamo allora l'operatore $\star$ introdotto nelle dispense. In particolare, definiamo una variante di $\star$, che chiameremo $\overline{\star}$, e che è definito come segue:\\
    \\
    Se $f : D_1 \rightarrow (D_2 + \{error\} + \{<stop, \sigma>\})$ e $g : D_2 \rightarrow (D_3 + \{error\} + \{<stop, \sigma>\})$, allora
    \begin{equation*}
        f\ \overline{\star}\ g : D_1 \rightarrow (D_3 + \{error\} + \{<stop, \sigma>\})
    \end{equation*}
    è definita come segue:
    \begin{align*}
        f\ \overline{\star}\ g =\ &\lambda x.\ cases\ fx\ of\\
            &\tb{1} d : gd;\\
            &\tb{1} error : error;\\
            &\tb{1} <stop, \sigma> : <stop, \sigma>
        \\
        & endcases
    \end{align*}
    
    Si intuisce già che l'idea è quella di, non appena si incontra il comando \textbf{stop}, propagare una coppia composta dall'elemento \textit{stop} e da una memoria fino all'interpretazione del programma stesso, che restituirà quindi l'output in tale memoria.\\
    \\
    La semantica dei programmi diventa quindi:
    
    \boxedaligned{0.7}{
        \mathcal{P}\doublebracket{\mbox{\textbf{program}}\ c}in =\ & cases\ (\mathcal{C}\doublebracket{c}\rho_0(\lambda x.unused)[in/lin][nil/lout])\ of\\
            &\tb{1} \sigma: \sigma\ lout;\\
            &\tb{1} error: error;\\
            &\tb{1} <stop, \sigma>: \sigma\ lout\\
        & endcases
    }
    
    L'interpretazione del comando \textbf{stop} è quindi definita come segue:
    
    \boxedaligned{0.3}{
        \mathcal{C}\doublebracket{\mbox{\textbf{stop}}}\rho = \lambda\sigma. <stop, \sigma>
    }
    
    Risulta chiaro quindi che è necessario modificare il dominio semantico della funzione d'interpretazione dei comandi $\mathcal{C}$, aggiungendo il caso in cui l'interpretazione di un comando produce la coppia $<stop,\sigma>$, ovvero:
    
    \begin{equation*}
        \mathcal{C} : Com \rightarrow \mathbb{AMB} \rightarrow \mathbb{MEM} \rightarrow (\mathbb{MEM} + \{error\} + (\{stop\} \times \mathbb{MEM}))
    \end{equation*}
    
    In aggiunta a queste modifiche, per rendere il nuovo meccanismo funzionante, c'è bisogno di aggiungere la funzionalità di propagare \textit{stop} a tutto il sistema di interpretazione. Per fare ciò basterà sostituire l'operarore $\overline{\star}$ all'operatore $\star$ all'interno delle funzioni di interpretazione semantica dei comandi già visti. Tutti gli altri rimarranno invece il vecchio operatore $\star$ già definito. Non importa, infatti, andare a modificare l'interpretazione di espressioni e dichiarazioni: per quanto riguarda le espressioni, vediamo che esse non contengono in alcun modo comandi, quindi tantomeno \textbf{stop}, e quindi il loro comportamento rimane inalterato; si potrebbe pensare che le dichiarazioni diano problemi, in quanto si utilizza la funzione d'interpretazione dei comandi per definire l'interpretazione della dichiarazione di procedure, tuttavia nemmeno questo caso da problemi in quanto, se anche una procedura contenesse \textbf{stop} al suo interno, ogni volta che si invoca una procedura, il corpo della procedura viene sostituito all'invocazione stessa (con il parametro attuale sostituito al parametro formale) e quindi si ``trasforma'' in un semplice comando all'interno del blocco in cui è avvenuta l'invocazione, riconducendo il tutto a casi già noti.\\
    \\
    Per concludere l'esercizio, mostriamo che i comandi $c_1;\mbox{\textbf{stop}}$ e $c_1;\mbox{\textbf{stop}};c_2$ risultano semanticamente equivalenti secondo la semantica appena descritta. Si distinguono tre casi, a seconda che il comando $c_1$ contenga \textbf{stop} o meno e che generi un errore o meno.
    
    \begin{itemize}
        \item $c_1$ non contiene \textbf{stop} e produce la memoria $\sigma_1$:
            \begin{align*}
                \mathcal{C}\doublebracket{c_1;\mbox{\textbf{stop}}}\rho\sigma =\ & \mathcal{C}\doublebracket{c_1}\rho\sigma\ \overline{\star}\ (\lambda\sigma'.\mathcal{C}\doublebracket{\mbox{\textbf{stop}}}\rho\sigma')\\
                =\ & \sigma_1\ \overline{\star}\ (\lambda\sigma'.\mathcal{C}\doublebracket{\mbox{\textbf{stop}}}\rho\sigma')\\
                =\ & (\lambda\sigma'.(\lambda\sigma''.<stop,\sigma''>)\sigma')\sigma_1 &\mbox{il valore viene fatto passare da $\overline{\star}$}\\
                =\ & <stop,\sigma_1>
            \end{align*}
            \begin{align*}
                \mathcal{C}\doublebracket{c_1;\mbox{\textbf{stop}};c_2}\rho\sigma =\ & \mathcal{C}\doublebracket{c_1}\rho\sigma\ \overline{\star}\ (\lambda\sigma'.\mathcal{C}\doublebracket{\mbox{\textbf{stop}}}\rho\sigma')\ \overline{\star}\ (\lambda\overline{\sigma}.\mathcal{C}\doublebracket{c_2}\rho\overline{\sigma})\\
                =\ & \sigma_1\ \overline{\star}\ (\lambda\sigma'.\mathcal{C}\doublebracket{\mbox{\textbf{stop}}}\rho\sigma')\ \overline{\star}\ (\lambda\overline{\sigma}.\mathcal{C}\doublebracket{c_2}\rho\overline{\sigma})\\
                =\ & (\lambda\sigma'.(\lambda\sigma''.<stop,\sigma''>)\sigma')\sigma_1\ \overline{\star}\ (\lambda\overline{\sigma}.\mathcal{C}\doublebracket{c_2}\rho\overline{\sigma})\\
                =\ & <stop,\sigma_1> \overline{\star}\ (\lambda\overline{\sigma}.\mathcal{C}\doublebracket{c_2}\rho\overline{\sigma})\\
                =\ & <stop,\sigma_1> &\mbox{\textit{stop} viene propagato da $\overline{\star}$}
            \end{align*}
            
            Sfruttando le proprietà del nuovo operatore, abbiamo dimostrato che, dati un ambiente ed una memoria qualsiasi, i due comandi producono la stessa coppia, ovvero sono semanticamente equivalenti.\\
            In modo analogo vediamo il caso in cui $c_1$ contenga \textbf{stop}.
            
        \item $c_1$ contiene \textbf{stop} e produce la coppia $<stop, \sigma_1>$:
            \begin{align*}
                \mathcal{C}\doublebracket{c_1;\mbox{\textbf{stop}}}\rho\sigma =\ & \mathcal{C}\doublebracket{c_1}\rho\sigma\ \overline{\star}\ (\lambda\sigma'.\mathcal{C}\doublebracket{\mbox{\textbf{stop}}}\rho\sigma')\\
                =\ & <stop, \sigma_1> \overline{\star}\ (\lambda\sigma'.\mathcal{C}\doublebracket{\mbox{\textbf{stop}}}\rho\sigma')\\
                =\ & <stop,\sigma_1> &\mbox{\textit{stop} viene propagato da $\overline{\star}$}
            \end{align*}
            \begin{align*}
                \mathcal{C}\doublebracket{c_1;\mbox{\textbf{stop}};c_2}\rho\sigma =\ & \mathcal{C}\doublebracket{c_1}\rho\sigma\ \overline{\star}\ (\lambda\sigma'.\mathcal{C}\doublebracket{\mbox{\textbf{stop}}}\rho\sigma')\ \overline{\star}\ (\lambda\overline{\sigma}.\mathcal{C}\doublebracket{c_2}\rho\overline{\sigma})\\
                =\ & <stop,\sigma_1> \overline{\star}\ (\lambda\sigma'.\mathcal{C}\doublebracket{\mbox{\textbf{stop}}}\rho\sigma')\ \overline{\star}\ (\lambda\overline{\sigma}.\mathcal{C}\doublebracket{c_2}\rho\overline{\sigma})\\
                =\ & <stop,\sigma_1> \overline{\star}\ (\lambda\overline{\sigma}.\mathcal{C}\doublebracket{c_2}\rho\overline{\sigma}) &\mbox{\textit{stop} viene propagato da $\overline{\star}$}\\
                =\ & <stop,\sigma_1> &\mbox{\textit{stop} viene propagato da $\overline{\star}$}
            \end{align*}
            
            Anche in questo caso risultano semanticamente equivalenti.\\
            Rimane da far vedere che in caso d'errore, l'operatore $\overline{\star}$ si comporta esattamente come $\star$.
        \item $c_1$ produce $error$:
            \begin{align*}
                \mathcal{C}\doublebracket{c_1;\mbox{\textbf{stop}}}\rho\sigma =\ & \mathcal{C}\doublebracket{c_1}\rho\sigma\ \overline{\star}\ (\lambda\sigma'.\mathcal{C}\doublebracket{\mbox{\textbf{stop}}}\rho\sigma')\\
                =\ & error\ \overline{\star}\ (\lambda\sigma'.\mathcal{C}\doublebracket{\mbox{\textbf{stop}}}\rho\sigma')\\
                =\ & error &\mbox{\textit{error} viene propagato da $\overline{\star}$}
            \end{align*}
            \begin{align*}
                \mathcal{C}\doublebracket{c_1;\mbox{\textbf{stop}};c_2}\rho\sigma =\ & \mathcal{C}\doublebracket{c_1}\rho\sigma\ \overline{\star}\ (\lambda\sigma'.\mathcal{C}\doublebracket{\mbox{\textbf{stop}}}\rho\sigma')\ \overline{\star}\ (\lambda\overline{\sigma}.\mathcal{C}\doublebracket{c_2}\rho\overline{\sigma})\\
                =\ & error\ \overline{\star}\ (\lambda\sigma'.\mathcal{C}\doublebracket{\mbox{\textbf{stop}}}\rho\sigma')\ \overline{\star}\ (\lambda\overline{\sigma}.\mathcal{C}\doublebracket{c_2}\rho\overline{\sigma})\\
                =\ & error\ \overline{\star}\ (\lambda\overline{\sigma}.\mathcal{C}\doublebracket{c_2}\rho\overline{\sigma}) &\mbox{\textit{error} viene propagato da $\overline{\star}$}\\
                =\ & error\ &\mbox{\textit{error} viene propagato da $\overline{\star}$}
            \end{align*}
    \end{itemize}
    
    Abbiamo quindi dimostrato che i due comandi proposti, sotto la semantica descritta, risultano equivalenti, ovvero $c_1;\mbox{\textbf{stop}} \equiv c_1;\mbox{\textbf{stop}};c_2$.
