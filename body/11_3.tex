\section*{Esercizio 11.3}

    \enunciato{1}{
        Utilizzando la caratterizzazione delle simulazioni forti vista nell'Esercizio 11.1, si provi che $Id$ è una simulazione e che se $R$ ed $S$ sono simulazioni allora $R \cup S$ ed $RS$ sono simulazioni.
    }
    
    Dato un LTS $<Q,A,\rightarrow\ >$, una relazione $R \subseteq Q \times Q$ è una \textit{simulazione (forte)} se, dati $p,q\in Q$ t.c. $<p,q>\in R$, allora vale
    \begin{equation*}
        \forall a\in A, p' \in Q:\ \mbox{se}\ p\xrightarrow{a}p'\ \mbox{allora}\ \exists q'\in Q:\ q\xrightarrow{a}q'\ \mbox{e}\ <p',q'>\in R,
    \end{equation*}
    per definizione.\\
    \\
    Tuttavia non è facile dimostrare quanto richiesto utilizzando questa definizione, quindi si sfrutta la proposizione esposta nell'Esercizio 11.1, la quale afferma che la relazione $R$ è una simulazione (forte) se e solo se
    \begin{equation*}
        \forall a\in A:\ R^{-1}\xrightarrow{a} \ \subseteq \ \xrightarrow{a} R^{-1}.
    \end{equation*}
    
    Questo equivale a dire, per definizione di inclusione di relazioni, che $R$ è una simulazione se e solo se, $\forall p,q\in Q$ e $a\in A$,
    \begin{equation}
        p\ R^{-1} \xrightarrow{a} q \ \mbox{implica}\ p \xrightarrow{a} R^{-1}\ q.
        \label{eq:simulazione_alt}
    \end{equation}  
    Quindi dimostrare la (\ref{eq:simulazione_alt}) è equivalente a dimostrare che $R$ è una simulazione.\\ 
    \\
    Essendo sia $R$ (e quindi $R^{-1}$) che $\xrightarrow{a}$ delle relazioni binarie tra stati in $Q$, applicando la definizione di composizione di relazioni otteniamo che una coppia di stati $<p,q>$, con $p,q\in Q$, è messa in relazione da $R^{-1}\xrightarrow{a}$ se e solo se
    \begin{equation*}
        \exists r\in Q:\ p\ R^{-1}\ r \xrightarrow{a} q,
    \end{equation*}
    ovvero
    \begin{equation*}
        \exists r\in Q:\ (r\ R\ p)\ \mbox{e}\ (r \xrightarrow{a} q).
    \end{equation*}
    Analogamente per la seconda relazione composta, otteniamo che $<p,q>\in \xrightarrow{a}R^{-1}$ se e solo se
    \begin{equation*}
        \exists r'\in Q:\ (p \xrightarrow{a} r')\ \mbox{e}\ (q\ R\ r').
    \end{equation*}
    
    Risulta allora chiaro che dimostrare la (\ref{eq:simulazione_alt}) corrisponde a dimostrare che, $\forall p,q\in Q$ e $a\in A$,
    \begin{align}
        \exists r\in Q:\ &(r\ R\ p)\ \mbox{e}\ (r \xrightarrow{a} q) \notag\\
        &\mbox{implica} \label{eq:simulazione_alt2}\\
        \exists r'\in Q:\ &(p \xrightarrow{a} r')\ \mbox{e}\ (q\ R\ r'), \notag
    \end{align}
    con $r$ ed $r'$ non necessariamente lo stesso stato. Quindi dimostrare la (\ref{eq:simulazione_alt2}) equivale a dimostrare la (\ref{eq:simulazione_alt}), la quale equivale a dimostrare che $R$ è una simulazione.\\
    \\
    Dimostriamo allora la (\ref{eq:simulazione_alt2}) per i tre casi richiesti.
    
    \begin{itemize}
        \item $Id$ è una simulazione\\
            Per dimostrare che la (\ref{eq:simulazione_alt2}) vale quando $R=Id$, basta prendere $r=p$ ed $r'=q$. Infatti, per definizione d'identità, $p\ Id\ p$ e $q\ Id\ q$, ed assumendo, per ipotesi, che $p \xrightarrow{a} q$, si ottiene il membro destro della (\ref{eq:simulazione_alt2}). Inoltre osserviamo che $Id=Id^{-1}$. Quindi abbiamo dimostrato che $p\ Id^{-1}\xrightarrow{a} q$ implica $p \xrightarrow{a}Id^{-1}\ q$, per $p,q\in Q$ ed $a\in A$ qualsiasi, provando la (\ref{eq:simulazione_alt}).
        \item $R \cup S$ è una simulazione\\
            Siano $R$ ed $S$ due simulazioni, per ipotesi. Quindi la (\ref{eq:simulazione_alt2}) vale per $R$ ed $S$. Scrivendo la (\ref{eq:simulazione_alt2}) per $R \cup S$ si ha che dobbiamo dimostrare che
            \begin{equation*}
                \exists r\in Q:\ (r\ R \cup S\ p)\ \mbox{e}\ (r \xrightarrow{a} q)\ \mbox{implica}\ \exists r'\in Q:\ (p \xrightarrow{a} r')\ \mbox{e}\ (q\ R \cup S\ r').
            \end{equation*}
            Ricordiamo che un certo elemento $x$ appartiene ad un insieme unione $A \cup B$ quando $x$ appartiene ad $A$ oppure a $B$. Quindi, in questo caso, se nel membro sinistro abbiamo $<r,p>\in R$, basterà prendere $<q,r'>\in R$, per un certo $r'$ che sappiamo esistere e sappiamo verificare la (\ref{eq:simulazione_alt2}), per l'ipotesi che $R$ è una simulazione. Analogamente nel caso in cui $<r,p> \in S$. Questo compre tutti i casi possibili, per definizione di insieme unione, e quindi dimostra la tesi.
        \item $RS$ è una simulazione\\
            Come nel caso precedente, dobbiamo dimostrare che
            \begin{equation*}
                \exists r\in Q:\ (r\ RS\ p)\ \mbox{e}\ (r \xrightarrow{a} q)\ \mbox{implica}\ \exists r'\in Q:\ (p \xrightarrow{a} r')\ \mbox{e}\ (q\ RS\ r').
            \end{equation*}
            Per la definizione di composizione di funzioni, dire che $r\ RS\ p$ equivale a dire che $\exists k:\ r\ R\ k\ S\ p$. Da $(r\ R\ k)$ e dall'ipotesi che $R$ è una simulazione (e che $(r \xrightarrow{a} q)$), si deduce che esiste un $k'$ tale che $(k \xrightarrow{a} k')$ e $(q\ R\ k')$. Da $(k\ S\ p)$ e dall'ipotesi che $S$ è una simulazione (e da $(k \xrightarrow{a} k')$ dedotto prima) si deduce invece che esiste un $r'$ tale che $(p \xrightarrow{a} r')$ e $(k'\ S\ r')$.\\
            Mettendo adesso insieme $(q\ R\ k')$ e $(k'\ S\ r')$ otteniamo che $(q\ RS\ r')$, il quale, assieme a $(p \xrightarrow{a} r')$ ricavato prima, costituisce la dimostrazione della tesi.
    \end{itemize}
    
    Con questo si concludono i tre casi richiesti, per i quali abbiamo dimostrato che la (\ref{eq:simulazione_alt2}) vale e quindi che le relazioni richieste sono effettivamente delle simulazioni.
    