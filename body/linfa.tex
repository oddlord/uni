% !TeX root = ../report-smc.tex
% !TeX encoding = UTF-8
% !TeX spellcheck = en_GB

\section*{LINFA: Smart drugs restocking}

  \subsection*{Smart drugs restocking}

    The pharmaceutical logistic chain is a very complex system, as it includes several actors and critical points (such as elevated drug costs, need for transport at monitored temperature, chance of expiration of goods, stock management, irregularity of demands, several possible logistic approaches, etc), which render the problem hard to optimize without the proper tools for decision support. In this context, possible unavailability could lead to critical situations, sometime even catastrophic, as it wouldn't be possible any more to guarantee the correct execution of one or more healthcare protocol, thus affecting the patients' health. Furthermore, orders are typically carried out on a daily basis and in a manual fashion, without the support of any decision support system. All these reasons makes restocking schedule hard, thus leading to unproductive stocks and higher stocking costs.
    
    In this scenario fits the project \textit{LINFA} (i.e. \textit{Logistica INtelligent del FArmaco}, or \textit{Smart Drug Logistic}, from Italian), that aims to develop an IT system for support to processes of drugs logistic management, in the context of healthcare or local companies. LINFA aims to increase efficiency, effectiveness and predictability of the process of drugs and medical devices restocking, within healthcare structures, through methods of predictive analysis and optimization, advanced logistic techniques and tracking features through the use of RFID technologies or the integration of healthcare and administrative information stream.
    
    For sake of simplicity, the following assumptions will be made in order to build a model:
    
    \begin{itemize}
      \item a single ward is present;
      \item the ward has a fixed number of beds and a fixed maximum storage capacity;
      \item patients are indistinguishable;
      \item there is only one type of drug;
      \item patients can arrive through emergencies (i.e. according to a random variable) or scheduled examinations (i.e. according to a fixed constant);
      \item each day, patients can leave the ward with a certain probability;
      \item restock orders are issued at the end of each day and arrive immediately.
    \end{itemize}
    
    \question{Try to build a model of a ward taking into consideration the previous assumptions. Consider maximum 40 beds in the ward and a stock capacity of 40 drugs units. Model the restock orders to be either of 0, 10, 20, 30 or 40 drug's units and with a uniform distribution. Consider also that each day each patient consumes a drug's unit.}
    \answer{
      
    }
