% !TeX spellcheck = en_GB
\documentclass{article}
\usepackage{graphicx}
\graphicspath{{img/}}
\begin{document}
    \pagenumbering{gobble}
    
    \begin{center}
        \includegraphics[scale=0.2]{logo_unifi.jpg}\\[4cm]
        \textit{Research Project for Smart Computing PhD Programme}\\[0.3cm]
        {\Huge Model-based quantitative analysis for on-line diagnosis, prediction and action scheduling in partially observable systems}
    \end{center}
    
    \vfill
    
    \textbf{Applicant}: Tommaso \textsc{Papini}
    
    \clearpage
    
    \section*{Motivations}
    
        More and more every year we see all around the world an increasing use of \textit{smart systems}, i.e. systems that incorporate the functions of sensing, actuation and control in order to describe and analyse a situation and make decisions based on the available data in a predictive and adaptive manner (i.e. performing \textit{smart} actions).
        
        Smart systems can have very different natures, as they are used in many distinct fields. Smart systems range from sensor networks to smart buildings, from cyber-physical systems to smart cities.
        
        In this context, \textit{smart computing} is a multidisciplinary area that mainly aims to study
        
    \section*{Goals}
    
    
    \section*{Approach}
    
    \clearpage
    
	\nocite{*}
	\bibliographystyle{IEEEtran}
	\bibliography{bibliography}
\end{document}