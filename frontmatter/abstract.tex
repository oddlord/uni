\cleardoublepage
\pdfbookmark[1]{Abstract}{Abstract}
\chapter*{Abstract}

	Nel 2014 il \acr{CERN} ha celebrato i suoi \textit{60 anni} di attività.\\
	Dalla sua fondazione, nel lontano Settembre 1954, il \ac{CERN} è stato protagonista di una serie di scoperte e innovazioni in svariati settori scientifici, informatica e fisica in primis, e si è affermato come uno tra i più importanti centri di ricerca in Europa e nel mondo.\\
	\\
	Il \ac{CERN} è infatti il più grande centro di ricerca di fisica nucleare in Europa, possessore, al momento, del più grande acceleratore di particelle al mondo: l'\acr{LHC}.\\
	Grazie al \ac{CERN} sono state fatte molte scoperte in ambito della fisica delle particelle, come la recente conferma sperimentale dell'effettiva esistenza del tanto cercato \textit{bosone di Higgs}: il 4 Luglio 2012 i team degli esperimenti \acr{ATLAS} e \acr{CMS} confermarono la scoperta del bosone di Higgs, scoperta che portò poi, il 10 Dicembre 2013, al conferimento del premio Nobel per la fisica a Peter Higgs e François Englert, principali ricercatori che teorizzarono l'esistenza di questa particella subatomica.\\
	\\
	Ma il \ac{CERN} è stato protagonista anche di molte innovazioni in ambito informatico: pensiamo al servizio internet Web, o \acr{WWW}, che fu ideato proprio al \ac{CERN} nel 1989 da Tim Berners-Lee, adesso fondatore e presidente del \acr{W3C}.\\
	\\
	È in questo contesto che si inquadra il progetto \textit{Improving the Worldwide Impact of Indico} che il sottoscritto, Tommaso Papini, ha seguito nel corso dei 14 mesi passati al \ac{CERN} con il programma di \textit{Technical Student} (dal 1 Ottobre 2013 al 30 Novembre 2014), sotto la supervisione di Pedro Ferreira (dipendente del \ac{CERN} ed attuale Project Manager di Indico) al \ac{CERN} e del Prof. Pierluigi Crescenzi (professore presso l'Università degli Studi di Firenze) da Firenze.\\
	\\
	Quest'elaborato, volto ad essere un resoconto di questi 14 mesi, sarà suddiviso in tre parti ed esporrà, in modo comprensivo ed esaustivo, il progetto seguito in Indico.\\
	All'interno della prima parte verrà esposta una panoramica dell'ambiente \ac{CERN} e del progetto Indico.\\
	Nella seconda parte, invece, saranno esposte le varie parti e le varie fasi che hanno composto il progetto stesso nell'arco di questi 14 mesi, assieme ad una panoramica sugli strumenti ed i linguaggi utilizzati.\\
	Infine, nella parte conclusiva, si parlerà di altri progetti minori seguiti al \ac{CERN} e di cosa si è ottenuto con questo progetto.
	