% !TeX spellcheck = it_IT
\section*{Esercizio 1: Simulatore di chiamate a procedura}

	\subsection*{Descrizione ad alto livello}
	
		Di seguito proponiamo lo pseudocodice relativo al primo esercizio.
		
        \begin{center}
           	\begin{lstlisting}[language=pseudo, gobble=14]
                analyze(start, end, depth){
                    char = load_char(start)
    	           	
                    switch(char){
                        case 's':
                            char2 = load_char(start+2)
                            switch(char2):{
                                case 'm':
                                    res = sum(start, end, depth)
                                default:
                                    res = sub(start, end, depth)
                            }
                        case 'p':
                            res = prod(start, end, depth)
                        case 'd':
                            res = div(start, end, depth)
                        default:
                            res = 0
                            while(start < end){
                                digit = load_char(start) - 48
                                res = res + digit
                                res = res * 10
                                start = start + 1
                            }
                            digit = load_char(start) - 48
                            res = res + digit
                    }
                    
                    return res
                }
               	
                main(){
                    file_descriptor = open("chiamate.txt")
                    buffer_pointer, length = read(file_descriptor)
                    close(file_descriptor)
    	           	
                    start = buffer_pointer + 1
                    end = buffer_pointer + length - 2
                    depth = 0
    	           	
                    analyze(start, end, depth)
    	           	
                    exit
                }
           	\end{lstlisting}
        \end{center}
		
		