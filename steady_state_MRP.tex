\documentclass[pdflatex, 11pt]{beamer}

\usepackage[utf8]{inputenc}
\usepackage[T1]{fontenc}
\usepackage[italian]{babel}
\usepackage{calligra}
\usepackage{graphicx}
\usepackage{tikz}
\usepackage{epstopdf}
\usepackage{spot}
\usepackage{multirow}
\usepackage{algorithm}% http://ctan.org/pkg/algorithms
\usepackage{algpseudocode}% http://ctan.org/pkg/algorithmicx

\graphicspath{{img/}}

\definecolor{links}{HTML}{2A1B81}
\hypersetup{colorlinks,linkcolor=links,urlcolor=links}

\mode<presentation>{
  \usetheme{Boadilla}
  \usecolortheme{beaver}
  \setbeamercovered{transparent=50}
  \setbeamertemplate{navigation symbols}{}
  \setbeamertemplate{itemize item}{\checkmark}
  \setbeamertemplate{itemize subitem}{-}
  \setbeamertemplate{enumerate items}[default]
}

\usebackgroundtemplate{
  \begin{tikzpicture}
    \node[opacity=0.05] {\includegraphics[scale=1]{img/logoUnifi.eps}};
  \end{tikzpicture}
}

\setspotlightstyle{rectangle}

\newcommand{\var}[1]{{\ttfamily#1}}% variable

\newenvironment<>{varblock}[2][\textwidth]{%
	\setlength{\textwidth}{#1}
	\begin{actionenv}#3%
		\def\insertblocktitle{#2}%
		\par%
		\usebeamertemplate{block begin}}
	{\par%
		\usebeamertemplate{block end}%
	\end{actionenv}}


\begin{document}

	\title[Motif Finding]{Analisi di algoritmi per il Motif Finding}
	\date[11 Dicembre 2015]{\flushright 11 Dicembre 2015}
	\institute[Università di Firenze]{\includegraphics[width=5cm]{img/logoUnifiName.eps}}
	
	\author[Papini - Bani]{
		\begin{center}
			\begin{tabular}{lr}
				Tommaso \textsc{Papini}&Gabriele \textsc{Bani}\\
				\href{mailto:tommaso.papini1@stud.unifi.it}{tommaso.papini1@stud.unifi.it}&
				\href{mailto:gabriele.bani@stud.unifi.it}{gabriele.bani@stud.unifi.it}
			\end{tabular}
		\end{center}
	}
	
	\titlegraphic{
		\vspace{-0.5cm}
		\tiny
		\href{http://creativecommons.org/licenses/by-sa/4.0/}{\includegraphics[width=1cm]{img/logoCC.png}}
		This work is licensed under a
		\href{http://creativecommons.org/licenses/by-sa/4.0/}{Creative Commons Attribution-ShareAlike 4.0 International License}.
	}
	
	\begin{frame}[plain]
		\titlepage
	\end{frame}
	
	\begin{frame}{Un po' di background}
		DNA:
		\begin{itemize}
			\item sequenza di nucleotidi
			\item 4 tipi di nucleotide: A, T, C, G
			\item \textit{l-mer}: sottosequenza di DNA di lunghezza $l$
		\end{itemize}
		\begin{block}{Motifs}
			In biologia può essere necessario ricavare certe sequenze di DNA ``nascoste''
			\begin{itemize}
				\item pattern di nucleotidi ripetuti (l-mer)
				\item utili a capire determinati comportamenti biologici
				\begin{itemize}
					\item sequenze di attivazione di geni specifici
				\end{itemize}
			\end{itemize}
		\end{block}
	\end{frame}
	
	\begin{frame}{Il problema del Motif Finding}
		Il problema del Motif Finding consiste nel ricavare un set di $t$ \textbf{l-mer} da un insieme di $t$ sequenze di DNA.
		\begin{block}{Input}
			\begin{itemize}
				\item $DNA$: matrice di nucleotidi $t \times n$
				\begin{itemize}
					\item $t$ sequenze di DNA
					\item ognuna di lunghezza $n$
				\end{itemize}
				\item $l$: lunghezza del motif cercato
			\end{itemize}
		\end{block}
		\begin{block}{Output}
			\begin{itemize}
				\item $s=(s_1,s_2,\dots,s_t)$: lista di $t$ posizioni iniziali di l-mer il più simili tra loro
			\end{itemize}
		\end{block}
	\end{frame}
	
	\begin{frame}{Un primo esempio}
		\begin{center}
			CGGGGCT\spot<2->[fill=yellow]{ATGCAACT}GGGTCGTCACATTCCCCTTTCGATA\\
			TTTGAGGGTGCCCAATAA\spot<2->[fill=yellow]{ATGCAACT}CCAAAGCGGACAAA\\
			GG\spot<2->[fill=yellow]{ATGCAACT}GATGCCGTTTGACGACCTAAATCAACGGCC\\
			AAGG\spot<2->[fill=yellow]{ATGCAACT}CCAGGAGCGCCTTTGCTGGTTCTACCTG\\
			AATTTTCTAAAAAGATTATAATGTCGGTCC\spot<2->[fill=yellow]{ATGCAACT}TC\\
			CTGCTGTACAACTGAGATCATGCTGC\spot<2->[fill=yellow]{ATGCAACT}TTCAAC\\
			TACATGATCTTTTG\spot<2->[fill=yellow]{ATGCAACT}TGGATGAGGGAATGATGC
		\end{center}
	\end{frame}
	
	\begin{frame}{Mutazioni random}
		\begin{center}
			CGGGGCT\spot<1->[fill=yellow]{ATcCAgCT}GGGTCGTCACATTCCCCTTTCGATA\\
			TTTGAGGGTGCCCAATAA\spot<1->[fill=yellow]{ggGCAACT}CCAAAGCGGACAAA\\
			GG\spot<1->[fill=yellow]{ATGgAtCT}GATGCCGTTTGACGACCTAAATCAACGGCC\\
			AAGG\spot<1->[fill=yellow]{AaGCAACc}CCAGGAGCGCCTTTGCTGGTTCTACCTG\\
			AATTTTCTAAAAAGATTATAATGTCGGTCC\spot<1->[fill=yellow]{tTGgAACT}TC\\
			CTGCTGTACAACTGAGATCATGCTGC\spot<1->[fill=yellow]{ATGCcAtT}TTCAAC\\
			TACATGATCTTTTG\spot<1->[fill=yellow]{ATGgcACT}TGGATGAGGGAATGATGC
		\end{center}
		Come trovare l'l-mer più simile tra tutti?
	\end{frame}
	
	\begin{frame}{Allineamento}
		\tiny\begin{align*}
			CGGGGCT&\spot<1->[fill=yellow]{ATcCAgCT}GGGTCGTCACATTCCCCTTTCGATA\\
			TTTGAGGGTGCCCAATAA&\spot<1->[fill=yellow]{ggGCAACT}CCAAAGCGGACAAA\\
			GG&\spot<1->[fill=yellow]{ATGgAtCT}GATGCCGTTTGACGACCTAAATCAACGGCC\\
			AAGG&\spot<1->[fill=yellow]{AaGCAACc}CCAGGAGCGCCTTTGCTGGTTCTACCTG\\
			AATTTTCTAAAAAGATTATAATGTCGGTCC&\spot<1->[fill=yellow]{tTGgAACT}TC\\
			CTGCTGTACAACTGAGATCATGCTGC&\spot<1->[fill=yellow]{ATGCcAtT}TTCAAC\\
			TACATGATCTTTTG&\spot<1->[fill=yellow]{ATGgcACT}TGGATGAGGGAATGATGC
		\end{align*}
	\end{frame}
	
	\begin{frame}{Profilo e Consenso}
		\begin{center}
			\begin{tabular}{l l l l l l l l l l}
				\multirow{8}{*}{\textbf{Allineamento}} & & A & T & C & C & A & G & C & T\\
				& & G & G & G & C & A & A & C & T\\
				& & A & T & G & G & A & T & C & T\\
				& & A & A & G & C & A & A & C & C\\
				& & T & T & G & G & A & A & C & T\\
				& & A & T & G & C & C & A & T & T\\
				& & A & T & G & G & C & A & C & T\\
				\hline
				\multirow{4}{*}{\textbf{Profilo}} & \textbf{A} & 5 & 1 & 0 & 0 & 5 & 5 & 0 & 0\\
				& \textbf{T} & 1 & 5 & 0 & 0 & 0 & 1 & 1 & 6\\
				& \textbf{G} & 1 & 1 & 6 & 3 & 0 & 1 & 0 & 0\\
				& \textbf{C} & 0 & 0 & 1 & 4 & 2 & 0 & 6 & 1\\
				\hline
				\textbf{Consenso} & & A & T & G & C & A & A & C & T 
			\end{tabular}
		\end{center}
	\end{frame}
	
	\begin{frame}{Score}
		Come definire la ``bontà'' di un set di l-mer?
		\begin{block}{Funzione score}
			Si definisce una funzione score sul vettore $s=(s_1,s_2,\dots,s_t)$ di posizioni iniziali:
			\begin{equation*}
				Score(s,DNA)=\sum_{j=1}^{l}M_{P(s)}(j)
			\end{equation*}
			dove
			\begin{itemize}
				\item $P(s)$: matrice profilo su $s$
				\item $M_{P(s)}(j)$: elemento massimo nella colonna $j$-esima di $P(s)$
			\end{itemize}
		\end{block}
		Si cerca il set di posizioni iniziali $s$ che massimizzi $Score(s,DNA)$!
	\end{frame}
	
	\begin{frame}{Score: l'esempio di prima}
		\begin{center}
			\begin{tabular}{l l l l l l l l l l}
				\multirow{8}{*}{\textbf{Allineamento}} & & A & T & C & C & A & G & C & T\\
				& & G & G & G & C & A & A & C & T\\
				& & A & T & G & G & A & T & C & T\\
				& & A & A & G & C & A & A & C & C\\
				& & T & T & G & G & A & A & C & T\\
				& & A & T & G & C & C & A & T & T\\
				& & A & T & G & G & C & A & C & T\\
				\hline
				\multirow{4}{*}{\textbf{Profilo}} & \textbf{A} & 5 & 1 & 0 & 0 & 5 & 5 & 0 & 0\\
				& \textbf{T} & 1 & 5 & 0 & 0 & 0 & 1 & 1 & 6\\
				& \textbf{G} & 1 & 1 & 6 & 3 & 0 & 1 & 0 & 0\\
				& \textbf{C} & 0 & 0 & 1 & 4 & 2 & 0 & 6 & 1\\
				\hline
				\textbf{Consenso} & & A & T & G & C & A & A & C & T 
			\end{tabular}
		\end{center}
		$Score(s,DNA) = 5+5+6+4+5+5+6+6=42$
	\end{frame}
	
	\begin{frame}{Score}
		Quanto può valere lo score?
		\begin{equation*}
			Score(s,DNA) = \begin{cases}
				l\cdot t, & \mbox{nel caso migliore}\\
				\frac{l\cdot t}{4}, & \mbox{nel caso peggiore}
			\end{cases}
		\end{equation*}
		\begin{itemize}
			\item $lt$ corrisponde al caso in cui tutti gli l-mer sono identici
			\item $\frac{lt}{4}$ corrisponde al caso in cui gli l-mer siano diversi in tutte le posizioni
		\end{itemize}
	\end{frame}
	
	\begin{frame}{Algoritmi brute force}
		\begin{block}{Forza bruta}
			In informatica il metodo ``forza bruta'' (o ricerca esaustiva della soluzione) è un algoritmo di risoluzione di un problema dato che consiste nel verificare tutte le soluzioni teoricamente possibili fino a che si trova quella effettivamente corretta.
		\end{block}
	\end{frame}
	
	\begin{frame}{Simple motif search}
		\begin{block}{L'idea}
			Esamina tutte le possibili combinazioni delle posizioni di partenza $s$ e prendi quella con maggior \textit{Score}.
		\end{block}
		\begin{itemize}
			\item Posizioni iniziali: $s=(1,\dots,1)$
			\item Posizioni  finali: $s=((n-l+1),\dots,(n-l+1))$
		\end{itemize}
		Si utilizza il metodo $NextElement$ per passare da un elemento al successivo in ordine alfabetico.
	\end{frame}
	
	\begin{frame}{Simple motif search}{Pseudocodice}
		\begin{center}
		    \begin{algorithmic}[1]
			    \Procedure{SimpleMotifSearch}{$DNA,t,n,l$}
				    \State $s\gets (1,1,\dots,1)$
				    \State $bestScore\gets Score(s,DNA)$
				    \While{$true$}
				    	\State $s\gets NextElement(s,t,n-l+1)$
				    	\If{$Score(s,DNA)>bestScore$}
				    		\State $bestScore\gets Score(s,DNA)$
				    		\State $bestMotif\gets s$
				    	\EndIf
				    	\If{$s=(1,1,\dots,1)$}
				    		\State \textbf{return} $bestMotif$
				    	\EndIf
				    \EndWhile
			    \EndProcedure
		    \end{algorithmic}
	    \end{center}
	\end{frame}
	
	\begin{frame}{Simple motif search}{Complessità}
		Quante iterazioni fa l'algoritmo?
		\begin{itemize}
			\item vengono esaminate tutte le possibili combinazioni
			\begin{itemize}
				\item $(n-l+1)$ possibili scelte per ogni posizione iniziale
				\item $t$ posizioni iniziali (una per ogni sequenza)
			\end{itemize}
			\item $(n-l+1)^t$ possibili combinazioni di posizioni iniziali
		\end{itemize}
		Quanti passi per calcolare \textit{Score}?
		\begin{itemize}
			\item $tl$ per calcolare la matrice Profilo
			\item $4l$ per calcolare \textit{Score}
		\end{itemize}
		\begin{block}{Costo}
			$$\mathcal{O}(tln^t)$$
		\end{block}
	\end{frame}
	
	\begin{frame}{Branch \& bound}
		\begin{block}{L'idea}
			Ottimizza \textit{Simple Motif Search} evitando di analizzare sequenze non ottimali
			\begin{itemize}
				\item enumerazione delle sequenze tramite \alert{\textbf{alberi}}
				\item funzione di \textit{Score ottimistico}
				\begin{itemize}
					\item \textit{Score parziale} su un nodo interno
					\item \textit{Score ideale} per le restanti posizioni
				\end{itemize}
			\end{itemize}
		\end{block}
	\end{frame}
	
	\begin{frame}{Branch \& bound}{Un esempio}
		\begin{center}
			\includegraphics[scale=0.4]{albero.png}
		\end{center}
	\end{frame}
	
	\begin{frame}{Branch \& bound}{Un esempio}
		\begin{center}
			\includegraphics[scale=0.4]{albero2.png}
		\end{center}
	\end{frame}
	
	\begin{frame}{Branch \& bound}
		Si esegue una visita in profondità dell'albero
		\begin{itemize}
			\item calcola lo \textit{Score ottimistico} per ogni nodo
			\item scarta i sottoalberi con \textit{Score ottimistico} sub-ottimo
		\end{itemize}
		\begin{block}{Score ottimistico}
			$$\mbox{Score ottimistico}=Score(s,i,DNA)+(t-i)\cdot l$$
			\begin{itemize}
				\item $Score(s,i,DNA)$: \textit{Score parziale} relativo alle prime $i$ sequenze di DNA
				\item $(t-i)\cdot l$: \textit{Score parziale} delle restanti posizioni (supponendole identiche)
			\end{itemize}
		\end{block}
		\begin{itemize}
			\item $NextVertex$ per passare al prossimo vertice (DFS)
			\item $Skip$ per passare al prossimo vertice saltando il sottoalbero attuale
		\end{itemize}
	\end{frame}
	
	\begin{frame}{Branch \& bound}{Pseudocodice}
		\begin{center}\scriptsize
		    \begin{algorithmic}[1]
			    \Procedure{BranchAndBoundMotifSearch}{$DNA,t,n,l$}
				    \State $s\gets (1,1,\dots,1)$
				    \State $bestScore\gets 0$
				    \State $i\gets 1$
				    \While{$i>0$}
						\If{$i<t$}
							\State $optimisticScore\gets Score(s,i,DNA)+(t-i)l$
							\If{$optimisticScore<bestScore$}
								\State $(s,i)\gets Skip(s,i,(n-l+1))$
							\Else
								\State $(s,i)\gets NextVertex(s,i,t,(n-l+1))$
							\EndIf
						\Else
							\If{$Score(s,DNA)>bestScore$}
								\State $bestScore\gets Score(s,DNA)$
								\State $bestMotif\gets s$
							\EndIf
							\State $(s,i)\gets NextVertex(s,i,t,(n-l+1))$
						\EndIf
				    \EndWhile
				    \State \textbf{return} $bestMotif$
			    \EndProcedure
		    \end{algorithmic}
	    \end{center}
	\end{frame}
	
	\begin{frame}{Branch \& bound}{Complessità}
		Quante iterazioni?
		\begin{itemize}
			\item una per ogni nodo interno/foglia (caso pessimo)
			\begin{itemize}
				\item $N=\frac{(n-l+1)^t-1}{(n-l+1)-1}$ nodi interni
				\item $L=(n-l+1)^t$ foglie
			\end{itemize}
			\item $N+L$ passi totali
		\end{itemize}
		Quanto passi per calcolare \textit{Score}?
		\begin{itemize}
			\item come prima!
		\end{itemize}
		\begin{block}{Costo}
			$$\mathcal{O}(tln^t)$$
		\end{block}
		Come prima!
		\begin{itemize}
			\item più costoso nel caso pessimo
			\item conveniente se esegue tanti \textit{Skip}
		\end{itemize}
	\end{frame}
	
	\begin{frame}{Algoritmi greedy}
	\end{frame}
	
	\begin{frame}{Algoritmi randomizzati}
	\end{frame}
	
	\begin{frame}{Conclusioni}
	\end{frame}
	
	\begin{frame}
		\begin{center}
			\textbf{\calligra\Huge Fine.}\\
			\includegraphics[width=5cm]{img/ornament.eps}\\[1cm]
			\pause
			{\huge\calligra Domande?\pause{} Grazie!}
		\end{center}
	\end{frame}

\end{document}