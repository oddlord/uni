\chapter{Conclusioni} \label{chap:conclusioni}

    Concludiamo questa tesi con alcune riflessioni sul lavoro svolto.
    
    Il Progetto KT per Indico è stato sicuramente molto utile per il progetto Indico in quanto ha aggiunto molte funzionalità che prima mancavano e che certamente lo renderanno più visibile al mondo esterno, anziché esclusivamente all'interno del \ac{CERN} e dell'ambiente della fisica delle particelle.
    
    Grazie al progetto di Cloud Deployment è adesso possibile, tramite un semplice script, installare Indico su una macchina virtuale o installarlo, da remoto, su un server cloud. Inoltre, grazie allo script di gestione, è anche resa più facile all'utente la gestione dell'istanza di Indico installata su macchina virtuale, fornendo una maggior destrezza di utilizzo.
    
    Lo script fabric per le distribuzioni di Indico ha invece reso certamente la vita più facile al team di sviluppo di Indico al \ac{CERN} che adesso potranno, tramite l'invocazione di un solo comando, creare nuove distribuzioni di Indico e caricarle su un server o su GitHub in maniera completamente automatica.
    
    Il progetto di Instance Tracking si è rivelato essere il più importante tra tutti in quanto ha portato allo sviluppo di un'applicazione completamente nuova e indipendente da Indico: Cephalopod. Grazie a Cephalopod sarà adesso possibile tracciare e creare statistiche sulle istanze di Indico installate in tutto il mondo. Non solo, Cephalopod riveste un ruolo ancora più importante in quanto è utilizzabile non soltanto da Indico, ma anche da altre applicazioni web che intendono tracciare le proprie istanze, sia al \ac{CERN} che altrove.
    
    L'ultima fase del progetto, anche se non molto rilevante dal punto di vista implementativo, è invece stata molto utile in quanto ha stabilito le basi di sviluppo sul quale il team di Indico andrà a creare, in futuro, un nuovo tool per la creazione e la personalizzazione di conferenze che sarà basato sul sistema ibrido blocco-widget ed implementerà un'intuitiva interfaccia drag-and-drop.
    
    Concludendo, l'esperienza al \ac{CERN} è stata un'esperienza altamente formativa, sia dal punto di vista professionale, permettendo di studiare ed utilizzare molti strumenti e linguaggi nuovi, che umano. Infatti, oltre all'aspetto lavorativo, i 14 mesi passati a Ginevra hanno portato ad un arricchimento e ad una crescita sotto molti punti di vista, che trascendono il semplice ambito professionale.
