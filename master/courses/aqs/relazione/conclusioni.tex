\chapter*{Conclusioni}
\addcontentsline{toc}{chapter}{Conclusioni}
\markboth{\textsc{\uppercase{Conclusioni}}}{\textsc{\uppercase{Conclusioni}}}
Concludiamo adesso con alcune considerazioni sui tools, gli esperimenti e le analisi viste in questo lavoro.\\
\\
Il tool principale di questo lavoro, Snort, si è rivelato essere molto utile e molto potente: non soltanto permette una rapida esecuzione tramite una serie di semplici comandi, ma da anche la possibilità all'utente di poter personalizzare molti dei suoi aspetti salienti. Per questo progetto, ad esempio, è stata scritta da zero una sola regola Snort (TOO MUCH PING) in quanto l'obiettivo principale era l'analisi temporale dei risultati e non la detection di più attacchi possibile, ma viste le potenzialità offerte non è assurdo ammettere che questo tool possa essere benissimo utilizzato in qualsiasi ambito che abbia a che fare con una rete di sistemi.\\
Dall'analisi temporale si è visto che le tempistiche di dection, al variare della tipologia dell'attacco, sono comunque ampiamente sotto il secondo e le tipologie di attacco non rilevate riscontrate sono state essenzialmente due traceroute e nmap con l'opzione -sn. Questo osservando inoltre che Snort è stato utilizzato con la totalità delle regole di dection può dare una buona confidenza sulla sua efficienza. Inoltre è supportato da una vasta community che costantemente segnala bug, aiuta lo sviluppo del software e, soprattutto, condivide con il mondo le proprie regole Snort, fornendo all'utente finale una maggior copertura e possibilità di scelta per quanto riguarda la customizzazione di Snort sul proprio sistema.\\
L'altro tool, utilizzato nella fase di analisi, è lo strumento di Business Intelligence Pentaho Enterprise Edition: questo tool di BI ci ha permesso, in modo molto veloce, efficiente ed elegante, di organizzare e modificare i dati raccolti dagli esperimenti, tramite il tool Kettle di Data Integration, e successivamente di poter effettuare analisi OLAP su questi dati, tramite lo strumento Mondrian di Business Analytics.\\
\\
La parte relativa agli esperimenti ci ha permesso di poter esaminare più da vicino alcuni comandi, come nmap o traceroute, o veri e propri tool di penetration testing come Metasploit. Certo gli esperimenti eseguiti non sono particolarmente interessanti o complessi ma nemmeno vogliono esserlo: dal momento che gli obiettivi principali di questo lavoro sono l'analisi dei dati ottenuti dagli esperimenti e la corretta applicazione di una valida metodologia di testing, piuttosto che gli esperimenti stessi, si è scelto di eseguire esperimenti semplici ed efficaci, per poterci poi concentrare sulla fase di analisi.\\
\\
Infine l'analisi temporale dei risultati sperimentali ci ha permesso innanzitutto di osservare da diversi punti di vista i vari tool o protocolli coinvolti negli esperimenti, dandoci la possibilità di trarre deduzioni o avanzare ipotesi sull'effettivo comportamento di questi. Inoltre è stato personalmente gratificante poter sfruttare le conoscenze acquisite durante il corso di Data Warehousing, sia per quanto riguarda tutta la teoria sottostante all'analisi multidimensionale OLAP, sia riguardo all'utilizzo di strumenti come Pentaho, che ci hanno agevolato non poco tutta la fase di analisi dei file di log. 