\chapter*{Introduzione}
	\addcontentsline{toc}{chapter}{Introduzione}
	\markboth{\textsc{\uppercase{Introduzione}}}{\textsc{\uppercase{Introduzione}}}
	
	Nel mondo delle basi di dati sono essenziali le infrastrutture tramite le quali si possono organizzare e memorizzare i dati, poiché a seconda del contesto di utilizzo può essere necessaria una maggiore efficienza, una maggiore ottimizzazione dello spazio o una maggiore flessibilità. In ogni caso il punto focale della ricerca in tale direzione rimane comunque l'oggetto di interesse, ovvero i dati stessi, e dove poterli reperire per poter popolare le strutture ottimizzate in base ai requisiti di utilizzo. Spesso capita che se un qualsiasi ente o azienda decide di orientare le proprie risorse nella creazione e nell'utilizzo di una base di dati, il reperimento degli stessi è a carico dell'azienda, come anche l'utilizzo e la fruizione: in generale, la proprietà del dato e tutti i diritti inerenti sono e rimangono dell'azienda che li ha generati o raccolti. Come è facile immaginare questo tipo di approccio può favorire l'azienda in termini di preferenza dell'utente in base alla qualità dei dati, ma limita enormemente il progresso ed il raffinamento dei dati, in quanto un'eventuale altra azienda (o un qualsiasi privato) che volesse ampliare o migliorare la qualità o il numero dei dati contenuti nella base di dati dovrebbe ripetere il passaggio (tipicamente oneroso) del reperimento delle informazioni, disperdendo tempo e risorse a spese del progresso innovativo.\\
	Per questo motivo nasce l'approccio \textit{Open Data}, che ha il proposito di rendere disponibili dati ``pubblici'' senza introdurre restrizioni sull'accesso o sull'utilizzo degli stessi. La migliore definizione di \textit{dato aperto} risulta essere:\\
	\begin{center}
		\textit{Un contenuto o un dato si definisce \textbf{aperto} se chiunque è in grado di utilizzarlo, riutilizzarlo e ridistribuirlo, soggetto, al massimo, alla richiesta di attribuzione e condivisione allo stesso modo.}
	\end{center}
	\vspace{0.5cm}
	Da qualche anno il Comune di Firenze ha adottato l'approccio \textit{open data} rendendo disponibile sul sito del Comune una collezione di 431 \textit{dataset} (ad oggi, \today) che possono essere consultati ed utilizzati liberamente da qualsiasi utente vi acceda. Allo scopo di incentivare il contributo degli utenti alle informazioni il Comune ha reso disponibile una grande varietà di \textit{dataset} che solo in minima parte risultano essere completi da un punto di vista quantitativo.\\
	Il proposito di questo elaborato consiste nell'organizzazione dei \textit{dataset} relativi al bilancio comunale degli anno 2012 e 2013 in un unico modello multidimensionale, consentendo quindi tutta una serie di analisi su di essi, come ad esempio il calcolo delle spese totali compiute nei due anni, il calcolo delle spese compiute dai vari settori comunali e come queste si siano modificate nei due anni, la possibilità di ricavare quali sono i settori più o meno costosi o di ricavare a quali aziende sono stati assegnati i lavori più lunghi. Questi sono soltanto alcuni esempi delle analisi condotte, come ultimo passo di questo progetto, all'interno del Capitolo \ref{chap:analisi}.\\
	Tra i due \textit{dataset} vi è una stretta relazione temporale (il secondo segue il primo) ed hanno la stessa struttura. Occorre tuttavia alterare in parte la loro struttura includendo nel \textit{dataset} finale ulteriori dati ed informazioni derivate; alla fase preparatoria segue una fase di modellazione concettuale e l'organizzazione dei dataset in un unico modello di Data Warehouse multidimensionale; infine la struttura potrà essere alimentata e successivamente interrogata tramite una serie di \textit{query} (richieste) che risulterebbero, in linea di massima, piuttosto ``scomode'' (in termini di efficienza computazionale) sui \textit{dataset} originali.\\
	Oltre ai classici strumenti di modellazione concettuale viene sperimentato l'approccio al problema tramite il framework di \textit{Business Intelligence} \textbf{Pentaho} che, oltre alle funzionalità di \textit{Data Mining}, \textit{analisi OLAP} ed utilizzo di metadati per la gestione dei \textit{Big Data}, consente anche di alimentare agilmente la struttura modellata fornendo un tool di \textit{Data Integration}.
