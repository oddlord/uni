\cleardoublepage
\pdfbookmark[1]{Abstract}{Abstract}
\chapter*{Abstract}

	Da anni le compagnie di Google e Yahoo dominano il settore della posta elettronica con i loro servizi di Gmail e Yahoo Mail. Rappresentano infatti i rispettivi rivali (assieme ad Hotmail) ed insieme vantano la maggior parte di utenti iscritti per il servizio di posta elettronica.\\
	\\
	Con lo sviluppo sempre maggiore del Web queste compagnie, come molte altre, hanno adottato la soluzione della Webmail, che consiste nel fornire il proprio servizio di posta elettronica direttamente dal Web. In questo modo si potrà accedere alla propria casella di posta soltanto con un browser e le proprie credenziali, senza il bisogno di configurare ogni volta un programma di posta elettronica.\\
	\\
	I servizi di Webmail di Google e Yahoo sono quindi vere e proprie applicazioni Web, dotate di determinate funzionalità e di un'interfaccia utente.\\
	\\
	L'obiettivo di quest'elaborato è proprio quello di analizzare, valutare e comparare le interfacce di questi due servizi di posta elettronica.\\
	All'interno del primo capitolo valuteremo l'usabilità di Gmail e Yahoo Mail seguendo il metodo della valutazione euristica proposto da Jakob Nielsen.\\
	Nel secondo capitolo, invece, costruiremo un modello di qualità per questi due siti web, come proposto da Roberto Polillo, utilizzando anche la valutazione di usabilità fatta nel capitolo precedente.
	