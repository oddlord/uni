% !TeX root = ../aal-modelos-transparencias.tex
% !TeX encoding = UTF-8
% !TeX spellcheck = es_ES

%********************************************************************
% Packages
%********************************************************************

\usepackage[spanish]{babel}
\usepackage[utf8]{inputenc}
\usepackage[T1]{fontenc}

\usepackage{amsmath}
\usepackage{amsfonts}
\usepackage{amssymb}
\usepackage{calligra}
\usepackage{graphicx}
\usepackage{tikz}
\usetikzlibrary{arrows,automata,shapes,calc,backgrounds,positioning}
\usepackage{times}

\graphicspath{{img/}}

%********************************************************************
% Beamer styles
%********************************************************************

\usetheme{Montpellier}
\usecolortheme{default}
\setbeamercovered{dynamic}
\newcommand{\nologo}{\setbeamertemplate{logo}{}} % command to set the logo to nothing

\setbeamertemplate{navigation symbols}{}
\setbeamertemplate{footline}[frame number]

\setbeamertemplate{footline}[frame number]
\setbeamertemplate{navigation symbols}{}

\usebackgroundtemplate{
  \begin{tikzpicture}
    \node[opacity=0.05] {\includegraphics[]{logoUnifi.png}};
  \end{tikzpicture}
}

\AtBeginSection[]{
  \begin{frame}
  \vfill
  \centering
  \begin{beamercolorbox}[sep=8pt,center,shadow=true,rounded=true]{title}
    \usebeamerfont{title}\insertsectionhead\par%
  \end{beamercolorbox}
  \vfill
  \end{frame}
}


\makeatletter
    \patchcmd{\chapter}{\if@openright\cleardoublepage\else\clearpage\fi}{}{}{}
\makeatother

\acrodef{CERN}{Conseil Européen pour la Recherche Nucléaire}

\begin{document}
    
    	\frenchspacing
    	\raggedbottom
    	\pagenumbering{gobble}
    	
    	\begin{center}
    		\large
    		\begingroup
    			\spacedallcaps{\myUni}\\
    			\myFaculty \\
    			\myDegree \\
    			\myTime
    		\endgroup \\
    		\vspace{0.5cm}
    		\begingroup
    			\color{Maroon}\spacedallcaps{\myTitle}\\
    			\spacedlowsmallcaps{\mySubtitle}\\
    			\bigskip
    		\endgroup
    	\end{center}
       	\begin{minipage}{0.5\textwidth}
       		\begin{flushleft}
       			\large \underline{Laureando:}\\
       			\normalsize Tommaso \textsc{Papini}\\
       			tommaso.papini1@stud.unifi.it
       		\end{flushleft}
       	\end{minipage}
       	\begin{minipage}{0.5\textwidth}
       		\begin{flushright}
       			\large \underline{Relatore:}\\
       			\normalsize Pierluigi \textsc{Crescenzi}\\
       			pierluigi.crescenzi@unifi.it
       		\end{flushright}
       	\end{minipage}
    
        {\let\clearpage\relax
            \chapter*{Riassunto Tesi di Laurea}
                Questa tesi è volta ad essere un resoconto dei 14 mesi passati al \acr{CERN}, Ginevra, tramite il programma di Techical Student.
                
                Il \ac{CERN} è un laboratorio europeo per la ricerca sul nucleare, istituito circa 10 anni fa. Grazie al \ac{CERN} sono state fatte molte scoperte in ambito fisico, come la scoperta del bosone di Higgs, ma anche molte innovazioni a livello informatico, come ad esempio la creazione del Web, da parte di Tim Berners-Lee.
                
                Il \ac{CERN} offre una vasta gamma di contratti e programmi, come ad esempio il Techical Student, oggetto di questa tesi, durato un totale di 14 mesi, da Ottobre 2013 a Novembre 2014.
                
                Durante questi 14 mesi si è lavorato all'interno del progetto Indico, ovvero una web application open source per la gestione di eventi sviluppata al \ac{CERN}, seguendo in particolare il Progetto Indico KT, volto a migliorare la visibilità a livello mondiale di Indico, a renderne più facile l'utilizzo e a rendere Indico visivamente più moderno.
                
                In particolare i principali progetti seguiti sono stati quattro: il \textit{Cloud Deployment} di Indico, ovvero l'automatizzazione dell'installazione di Indico in ambiente cloud e in ambiente virtuale; \textit{Distribuzione e Packaging}, ovvero l'automatizzazione della creazione di distribuzioni di Indico e della loro distribuzione; l'\textit{Instance Tracking}, per il quale è stata scritta un applicazione che traccia le istanze di Indico nel mondo e crea delle statistiche sui dati raccolti; infine la \textit{Conference Customization}, volta ad esplorare le possibilità per un possibile aggiornamento futuro allo strumento di personalizzazione delle conferenze in Indico.
                
                I linguaggi, gli strumenti e le tecnologie utilizzate per portare a termine i vari progetti sono stati svariati. In particolare i linguaggi predominanti sono stati Python, Javascript, HTML5 e CSS. Gli principali ambiti dei vari progetti sono programmazione web e scripting (Python e Bash).
        }
\end{document}