% !TeX root = ../assembly-lines-analysis-presentation.tex
% !TeX encoding = UTF-8
% !TeX spellcheck = en_GB

\section{Analysis of assembly lines}

  \begin{frame}{Why analyse assembly lines?}
    \begin{itemize}
      \setlength\itemsep{1em}
      \item Maximize throughput and efficiency
      \begin{itemize}
        \item speeding-up/slowing-down workstations in order to balance throughput
        \item identify bottlenecks and focus resources on them
      \end{itemize}
      \item Minimize resources and energy consumption
      \begin{itemize}
        \item lower power consumption before bottlenecks
      \end{itemize}
      \item Adapt production during runtime
      \begin{itemize}
        \item Industrie 4.0
        \item e.g. schedule personnel/resources in a tactic perspective
      \end{itemize}
    \end{itemize}
  \end{frame}
  
  \subsection{Assembly lines}
    \begin{frame}{Assembly line}
      \begin{center}\scalebox{0.9}{\tikzstyle{block} = [rectangle, draw, rounded corners, minimum height=2.5em, align=center, fill=white]
\tikzstyle{line} = [draw, -latex', line width=.3mm]

\begin{tikzpicture}[node distance = 1.5em, auto]
  % Place nodes
  \node [text width=1cm, align=center] at (0,0) (arrival) {\textit{product\\arrival}};
  \node [right = of arrival, block, square] (ws1) {$\pl{WS}_1$};
  \node [right = of ws1, block, square] (ws2) {$\pl{WS}_2$};
  \node [right = of ws2] (dots1) {\ldots};
  \node [right = of dots1, block, square] (wsk) {$\pl{WS}_{k}$};
  \node [right = of wsk] (dots2) {\ldots};
  \node [right = of dots2, block, square] (wsN) {$\pl{WS}_{N}$};
  \node [right = of wsN, text width=1.5cm, align=center] (production) {\textit{production\\complete}};
  % Draw edges
  \path [line] (arrival) -- (ws1);
  \path [line] (ws1) -- (ws2);
  \path [line] (ws2) -- (dots1);
  \path [line] (dots1) -- (wsk);
  \path [line] (wsk) -- (dots2);
  \path [line] (dots2) -- (wsN);
  \path [line] (wsN) -- (production);
\end{tikzpicture}}\end{center}
      
      \vspace{1em}
      \begin{minipage}{0.6\textwidth}
        $N$ sequential workstations $\pl{WS}_1, \dots, \pl{WS}_N$
        \begin{itemize}
          \item with transfer blocking
          \item and no buffering capacity
        \end{itemize}
        
        \vspace{1em}
        Workstation $\pl{WS_k}$ can be in one of three states
        \begin{itemize}
          \item \textit{producing}: $\pl{WS_k}$ is working on a product
          \item \textit{done}: $\pl{WS_k}$ is done working on a product
          \item \textit{idling}: $\pl{WS_k}$ is waiting for a new product
        \end{itemize}
      \end{minipage}
      \begin{minipage}{0.35\textwidth}
        \begin{center}\scalebox{0.8}{%Components styles
\tikzstyle{component state}=[draw, ellipse, minimum height = 0.5cm, minimum width = 2cm, thick, fill=white]
\tikzstyle{arc label}=[draw, align=left, fill=white, font={\scriptsize, \itshape}]
\tikzstyle{line} = [draw, -latex']

\begin{tikzpicture}
	%Workstation
	\node [component state] at (5,0) (WIdle) {idling};
	\node [component state] at (8,0) (WProducing) {producing};
	\node [component state] at (6.5,-2) (WDone) {done};
	
	\path [black,line,out=45,in=135] (WIdle) edge (WProducing);
	\path [black,line,out=270,in=0] (WProducing) edge (WDone);
	\path [black,line,out=180,in=270] (WDone) edge (WIdle);
	
	\node[arc label, align=center] at (6.5,1) {a product has arrived\\from workstation $\pl{WS}_{k-1}$};
	\node[arc label, align=center] at (8,-1.1) {production\\has finished};
	\node[arc label, align=center] at (5,-1.1) {workstation $\pl{WS}_{k+1}$\\has received\\the product};
\end{tikzpicture}}\end{center}
    \end{minipage}
    \end{frame}
    
    \begin{frame}{Workstation}
      Each workstation $\pl{WS}_k$
      \begin{itemize}
        \item has no internal parallelism
        \begin{itemize}
          \item at most one item being processed in each workstation
        \end{itemize}
        \item can implement complex workflows
        \begin{itemize}
          \item sequential/alternative/cyclic phases with random choices
        \end{itemize}
        \item and has GEN phases' durations
      \end{itemize}
      
      \begin{center}\scalebox{0.7}{\tikzstyle{decision} = [diamond, draw, node distance=3cm, inner sep=0pt, minimum height=3em, minimum width=3em, align=center, fill=white]
\tikzstyle{block} = [rectangle, draw, rounded corners, minimum height=2.5em, align=center, fill=white]
\tikzstyle{line} = [draw, -latex', line width=.3mm]

\begin{tikzpicture}[node distance = 1.5em, auto]
  \draw [black!25, line width=.5mm, rounded corners=1.75ex, fill=black!5] (0,0) rectangle (10.2,4) node[fitting node] (ws) {};
  \node [right = 2.5em of ws.north west, rectangle, draw, color=black!25, line width=.5mm, fill=white, text=black, rounded corners, minimum width=4em, minimum height=2em] (ws-name) {$\pl{WS}_k$};
  % Place nodes
  \node [above right = 6em and 1.5em of ws.south west, block, square] (p1) {$\pl{p}_1$};
  \node [left = 5em of p1] (wsk-1) {};
  \node [right = of p1, block, square] (p2) {$\pl{p}_2$};
  \node [right = of p2, decision] (s1) {$\pl{s}_1$};
  \node [above right = of s1, block, square] (p3a) {$\pl{p}_{3a}$};
  \node [below right = of s1, block, square] (p3b) {$\pl{p}_{3b}$};
  \node [right = of p3a, block, square] (p4a) {$\pl{p}_{4a}$};
  \node [right = of p3b, block, square] (p4b) {$\pl{p}_{4b}$};
  \node [right = of p4b, decision] (s2) {$\pl{s}_2$};
  \node [above right = of s2, block, square] (p5) {$\pl{p}_5$};
  \node [right = of p5, block, square] (d) {$\pl{d}$};
  \node [right = 5em of d] (wsk+1) {};
  % Draw edges
  \path [line] (wsk-1) -- (p1);
  \path [line] (p1) -- (p2);
  \path [line] (p2) -- (s1);
  \path [line] (s1) |- (p3a);
  \path [line] (s1) |- (p3b);
  \path [line] (p3a) -- (p4a);
  \path [line] (p3b) -- (p4b);
  \path [line] (p4b) -- (s2);
  \path [line] (s2) -- ++(0,-0.8) -| (p3b);
  \path [line] (p4a) -| (p5);
  \path [line] (s2) -| (p5);
  \path [line] (p5) -- (d);
  \path [line] (d) -- (wsk+1);
\end{tikzpicture}}\end{center}
      
      The last phase has no duration and encodes the \textit{done} state
    \end{frame}
    
    \begin{frame}{Underlying stochastic process}
      The underlying stochastic process of each isolated workstation\\
      is a Semi Markov Process (SMP)
      \begin{itemize}
        \item due to GEN durations
        \item and the absence of internal parallelism
      \end{itemize}
      
      \vspace{1.5em}
      The whole assembly line finds a renewal in any case where
      \begin{itemize}
        \item every \textit{done} station is in a queue before a bottleneck
        \item and everything else is \textit{idling}
      \end{itemize}
      
      \begin{center}\scalebox{0.8}{\relaxnewsetlength{\nodewidth}{2.5em}
\relaxnewsetlength{\nodedistance}{1.5em}
\relaxnewsetlength{\workingnodewidth}{0.7\nodewidth}

\tikzstyle{block} = [rectangle, draw, rounded corners, minimum height=\nodewidth, align=center, fill=white]
\tikzstyle{line} = [draw, -latex', line width=.3mm]
\definecolor{notidlegray}{gray}{0.8}
\tikzstyle{notidle} = [fill=notidlegray]

\begin{tikzpicture}[node distance = \nodedistance, auto]
  % Place nodes
  \node [text width=1cm, align=center] at (0,0) (arrival) {\textit{arrival}};
  \node [right = of arrival, block, square] (ws1) {$\pl{WS}_1$};
  \node [right = of ws1, block, square, notidle] (ws2) {$\pl{WS}_{2}$};
  \node [right = of ws2, block, square, notidle] (ws3) {$\pl{WS}_{3}$};
  \node [right = of ws3, block, square] (ws4-white-background) {};
  \fill [notidlegray, draw] ($(ws4-white-background.north west) + (\workingnodewidth,0)$) {[rounded corners] -- ++(-\workingnodewidth,0) -- ++(0,-\nodewidth)} -- ++(\workingnodewidth,0) -- cycle {};
  \node [right = of ws3, block, square, fill=none] (ws4) {$\pl{WS}_{4}$};
  \node [right = of ws4, block, square] (ws5) {$\pl{WS}_{5}$};
  \node [right = of ws5, block, square] (ws6) {$\pl{WS}_{6}$};
  \node [right = of ws6, text width=1.5cm, align=center] (production) {\textit{completion}};
  % Draw edges
  \path [line] (arrival) -- (ws1);
  \path [line] (ws1) -- (ws2);
  \path [line] (ws2) -- (ws3);
  \path [line] (ws3) -- (ws4);
  \path [line] (ws4) -- (ws5);
  \path [line] (ws5) -- (ws6);
  \path [line] (ws6) -- (production);
\end{tikzpicture}}\end{center}
    \end{frame}
  
  \subsection{Inspection}
    \begin{frame}{Inspection with partial observability}
      
      The assembly line can be inspected by external observers
      \begin{itemize}
        \item the line can be considered at steady-state at inspection
        \item there can be ambiguity about the current phase
      \end{itemize}
      
      \vspace{1em}
      An observation is a tuple $\omega=\langle \omega_0, \omega_1,\ldots, \omega_N \rangle$
      \begin{itemize}
        \item $\omega_0$ indicates if a new product is ready to enter the line or not
        \item $\omega_k = \langle \sigma_k, \phi_k \rangle$ refers to $\pl{WS}_k$
        \begin{itemize}
          \item $\sigma_k$ indicates if $\pl{WS}_k$ is \textit{idle}/\textit{producing}/\textit{done}
          \item $\phi_k$ identifies the set of possible current phases
        \end{itemize}
      \end{itemize}
        
      \vspace{1em}
      \begin{minipage}{0.65\textwidth}
        Two kinds of uncertainty
        \begin{itemize}
          \item about the actual current phase
          \begin{itemize}
            \item discrete
          \end{itemize}
          \item about the remaining time in the current phase%
          \begin{itemize}
            \item continuous
          \end{itemize}
        \end{itemize}
      \end{minipage}
      \begin{minipage}{0.3\textwidth}
        \begin{center}\scalebox{0.65}{\let\observationdistance\relax
\newlength{\observationdistance}
\setlength{\observationdistance}{0.1em}

\tikzstyle{decision} = [diamond, draw, node distance=3cm, inner sep=0pt, minimum height=3em, minimum width=3em, align=center, fill=white]
\tikzstyle{block} = [rectangle, draw, rounded corners, minimum height=2.5em, align=center, fill=white]
\tikzstyle{line} = [draw, -latex', line width=.3mm]

\tikzstyle{north-east-lines} = [preaction={fill, white}, pattern=north east lines, pattern color=black!30]

\begin{tikzpicture}[node distance = 1.5em, auto]
  \draw [black!25, line width=.5mm, rounded corners=1.75ex, fill=black!5] (0,0) rectangle (5.1,3.7) node[fitting node] (ws) {};
  \node [right = 1em of ws.north west, rectangle, draw, color=black!25, line width=.5mm, fill=white, text=black, rounded corners, minimum width=2em, minimum height=1.5em] (ws-name) {$\pl{WK}_{k}$};
  % Place nodes
  \node [above right = 4.3em and 1.5em of ws.south west, block, square, fill=black!25] (p1) {$\pl{p}_1$};
  \node [above = \observationdistance of p1] (p1-obs) {$o_1$};
  \node [left = 3em of p1] (wsk-1) {};
  \node [right = of p1, decision] (s1) {$\pl{s}_1$};
  \node [above right = of s1, block, square, north-east-lines] (p2a) {$\pl{p}_{2a}$};
  \node [above = \observationdistance of p2a] (p2a-obs) {$o_2$};
  \node [below right = of s1, block, square, north-east-lines] (p2b) {$\pl{p}_{2b}$};
  \node [above = \observationdistance of p2b] (p2b-obs) {$o_2$};
  \node [below right = of p2a, block, square] (d) {$\pl{d}$};
  \node [right = 3em of d] (wsk+1) {};
  % Draw edges
  \path [line] (wsk-1) -- (p1);
  \path [line] (p1) -- (s1);
  \path [line] (s1) |- (p2a);
  \path [line] (s1) |- (p2b);
  \path [line] (p2a) -| (d);
  \path [line] (p2b) -| (d);
  \path [line] (d) -- (wsk+1);
\end{tikzpicture}}\end{center}
      \end{minipage}
    \end{frame}
  
  \subsection{Performance measures}
    \begin{frame}{Performance measures}{Time To Done}
      \begin{center}
        The remaining time until workstation $k$,\\
        according to observation $\omega$, reaches the \textit{done} state
      \end{center}
      
      \vspace{1em}
      \begin{center}\scalebox{0.9}{%Components styles
\tikzstyle{component state}=[draw, ellipse, minimum height = 0.5cm, minimum width = 2cm, thick, fill=white]
\tikzstyle{arc label}=[draw, align=left, fill=white, font={\scriptsize, \itshape}]
\tikzstyle{line} = [draw, -latex']

\begin{tikzpicture}
	%Workstation
	\node [component state] at (5,0) (WIdle) {idling};
	\node [component state] at (8,0) (WProducing) {producing};
	\node [component state, fill=black!25] at (6.5,-2) (WDone) {done};
	
	\path [black,line,out=45,in=135] (WIdle) edge (WProducing);
	\path [black,line,out=270,in=0] (WProducing) edge (WDone);
	\path [black,line,out=180,in=270] (WDone) edge (WIdle);
	
	\node[arc label, align=center] at (6.5,1) {a product has arrived\\from workstation $\pl{WS}_{k-1}$};
	\node[arc label, align=center] at (8,-1.1) {production\\has finished};
	\node[arc label, align=center] at (5,-1.1) {workstation $\pl{WS}_{k+1}$\\has received\\the product};
\end{tikzpicture}}\end{center}
    \end{frame}
    
    \begin{frame}{Performance measures}{Time To Idle}
      \begin{center}
        The remaining time until workstation $k$,\\
        according to observation $\omega$, reaches the \textit{idling} state
      \end{center}
      
      \vspace{1em}
      \begin{center}\scalebox{0.9}{%Components styles
\tikzstyle{component state}=[draw, ellipse, minimum height = 0.5cm, minimum width = 2cm, thick, fill=white]
\tikzstyle{arc label}=[draw, align=left, fill=white, font={\scriptsize, \itshape}]
\tikzstyle{line} = [draw, -latex']

\begin{tikzpicture}
	%Workstation
	\node [component state, fill=black!25] at (5,0) (WIdle) {idling};
	\node [component state] at (8,0) (WProducing) {producing};
	\node [component state] at (6.5,-2) (WDone) {done};
	
	\path [black,line,out=45,in=135] (WIdle) edge (WProducing);
	\path [black,line,out=270,in=0] (WProducing) edge (WDone);
	\path [black,line,out=180,in=270] (WDone) edge (WIdle);
	
	\node[arc label, align=center] at (6.5,1) {a product has arrived\\from workstation $\pl{WS}_{k-1}$};
	\node[arc label, align=center] at (8,-1.1) {production\\has finished};
	\node[arc label, align=center] at (5,-1.1) {workstation $\pl{WS}_{k+1}$\\has received\\the product};
\end{tikzpicture}}\end{center}
    \end{frame}
    
    \begin{frame}{Performance measures}{Time To Start Next}
      \begin{center}
        The remaining time until workstation $k$,\\
        according to observation $\omega$, starts the production of a new product
      \end{center}
      
      \vspace{1em}
      \begin{center}\scalebox{0.9}{%Components styles
\tikzstyle{component state}=[draw, ellipse, minimum height = 0.5cm, minimum width = 2cm, thick, fill=white]
\tikzstyle{arc label}=[draw, align=left, fill=white, font={\scriptsize, \itshape}]
\tikzstyle{line} = [draw, -latex']

\begin{tikzpicture}
	%Workstation
	\node [component state] at (5,0) (WIdle) {idling};
	\node [component state] at (8,0) (WProducing) {producing};
	\node [component state] at (6.5,-2) (WDone) {done};
	
	\path [black,line,out=45,in=135] (WIdle) edge (WProducing);
	\path [black,line,out=270,in=0] (WProducing) edge (WDone);
	\path [black,line,out=180,in=270] (WDone) edge (WIdle);
	
	\node[arc label, align=center, fill=black!25] at (6.5,1) {a product has arrived\\from workstation $\pl{WS}_{k-1}$};
	\node[arc label, align=center] at (8,-1.1) {production\\has finished};
	\node[arc label, align=center] at (5,-1.1) {workstation $\pl{WS}_{k+1}$\\has received\\the product};
\end{tikzpicture}}\end{center}
    \end{frame}