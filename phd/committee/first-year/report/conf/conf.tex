% !TeX root = ../report-phd-first-year.tex
% !TeX encoding = UTF-8
% !TeX spellcheck = en_GB

%********************************************************************
% Packages
%********************************************************************

\usepackage[utf8]{inputenc}
\usepackage[T1]{fontenc}
\usepackage[english]{babel}
\usepackage[bottom, para]{footmisc}
\usepackage{amsmath,amssymb,amsthm}
\usepackage{graphicx}
\usepackage{hyperref}
\usepackage{perpage}
\usepackage{xcolor}
\usepackage{epstopdf}
\usepackage{placeins}
\usepackage[nolist,nohyperlinks]{acronym}

%********************************************************************
% Packages options
%********************************************************************

% geometry
\usepackage{geometry}
\geometry{
    a4paper,
    ignoremp,
    bindingoffset = 1cm, 
    textwidth     = 13.5cm,
    textheight    = 21.5cm,
    lmargin       = 1cm,
    rmargin		    = 3cm,
    tmargin       = 2cm,
    bmargin       = 3cm
}

% perpage
\MakePerPage{footnote}

% hyperref
\definecolor{webgreen}{rgb}{0,.5,0}
\definecolor{webbrown}{rgb}{.6,0,0}
\definecolor{RoyalBlue}{cmyk}{1, 0.50, 0, 0}
\hypersetup{%
  %draft, % = no hyperlinking at all (useful in b/w printouts)
  colorlinks=true, linktocpage=true, pdfstartpage=3, pdfstartview=FitV,%
  % uncomment the following line if you want to have black links (e.g., for printing)
  % colorlinks=false, linktocpage=false, pdfstartpage=3, pdfstartview=FitV, pdfborder={0 0 0},%
  breaklinks=true, pdfpagemode=UseNone, pageanchor=true, pdfpagemode=UseOutlines,%
  plainpages=false, bookmarksnumbered, bookmarksopen=true, bookmarksopenlevel=1,%
  hypertexnames=true, pdfhighlight=/O,%nesting=true,%frenchlinks,%
  urlcolor=RoyalBlue, linkcolor=black, citecolor=webbrown, %pagecolor=RoyalBlue,%
  %urlcolor=Black, linkcolor=Black, citecolor=Black, %pagecolor=Black,%
}

% graphicx
\graphicspath{{img/}}

%********************************************************************
% New commands
%********************************************************************

% Reversed \ac
% e.g. MRP (Markov Regenerative Process)
\makeatletter
  \newcommand*{\acrf}{\AC@starredfalse\protect\acrfa}%
  \WithSuffix\newcommand\acrf*{\AC@starredtrue\protect\acrfa}%
  \newcommand*{\acrfa}[1]{%
    \texorpdfstring{\protect\@acrf{#1}}{#1 (\AC@acl{#1})}
  }
  \newcommand*{\@acrf}[1]{%
    \ifAC@footnote
      \acsfont{\AC@acs{#1}}%
      \footnote{\AC@placelabel{#1}\AC@acl{#1}{}}%
    \else
      \acffont{%
        \acfsfont{\acsfont{\AC@acs{#1}}}%
        \nolinebreak[3] %
        (\AC@placelabel{#1}\AC@acl{#1})%
      }%
    \fi
    \ifAC@starred\else\AC@logged{#1}\fi
  }
  \newcommand{\@acr}[1]{%
    \ifAC@dua
      \ifAC@starred\acl*{#1}\else\acl{#1}\fi%
    \else
      \expandafter\ifx\csname AC@#1\endcsname\AC@used%
        \ifAC@starred\acs*{#1}\else\acs{#1}\fi%
      \else
        \ifAC@starred\acrf*{#1}\else\acrf{#1}\fi%
      \fi
    \fi
  }
  \newcommand*{\acr}{\AC@starredfalse\protect\@acr}%
  \WithSuffix\newcommand\acr*{\AC@starredtrue\protect\@acr}%
\makeatother

% Full acronym with plural long and singular short
% e.g. Markov Regenerative Processes (MRP)
\makeatletter
  \renewcommand*{\acfpa}[1]{%
    \texorpdfstring{\protect\@acfp{#1}}{\AC@aclp{#1} (\AC@acs{#1})}
  }
  \renewcommand*{\@acfp}[1]{%
    \ifAC@footnote
      \acsfont{\AC@acsp{#1}}%
      \footnote{\AC@placelabel{#1}\AC@aclp{#1}{}}%
    \else
      \acffont{%
        \AC@placelabel{#1}\AC@aclp{#1}%
        \nolinebreak[3] %
        \acfsfont{(\acsfont{\AC@acs{#1}})}%
      }%
    \fi
    \ifAC@starred\else\AC@logged{#1}\fi
  }
\makeatother
