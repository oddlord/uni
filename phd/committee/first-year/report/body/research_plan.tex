% !TeX root = ../report-phd-first-year.tex
% !TeX encoding = UTF-8
% !TeX spellcheck = en_GB

\section*{Research plan for the next year}

  One of the main topics that will be further investigated during the next year is the analysis of assembly line. At the moment, the work produced \cite{biagi2017inspection} sets a good base for investigation of analysis techniques for assembly lines, but lacks many aspects in order to be usable with real assembly lines. The first restriction that should be surpassed is the lack of buffers between workstations. In real assembly lines there is usually a buffer of fixed size, so one of the first investigations should be towards the introduction of buffering capacity, considering the same capacity for every workstation or even different sizes. More performance measures should also be introduced, such as the time until the production from the whole assembly line of a specific product in the line, or the time until the next $N$ products will be completed by the assembly line. These kind of measures would be complex than the ones already proposed in \cite{biagi2017inspection}, but would also allow a more accurate analysis of the assembly line. Evaluation of these additional measures should also be derived in a compositional fashion, as done already in \cite{biagi2017inspection}, in order to keep the technique computationally feasible. Lastly, a better upper bound for the measures proposed should be derived, similarly to the lower bound derived in \cite{biagi2017inspection}.
  
  Another topic that will be investigated is that of smart drugs restocking for the \acr{LINFA} project. The work conducted at the moment is still far from being usable in a real ward scenario. This mainly due the problem of state space explosion, as in order to model many aspects of the problem, a not optimised model would inevitably generate too many states, rendering it impossible to analyse. One of the first directions is thus that of optimising the prediction model itself, in order to model more aspects while maintaining the number of generated states feasible. Then, the restrictions currently present in the model should one by one be surpassed. An example would be to introduce different healthcare protocols for different kinds of patients, including different phases with different drugs consumption rates, thus personalising more the sojourn of patients in the ward. This could be done by employing techniques of process mining \cite{van2004workflow}, allowing to generate models automatically from historical data, such as a dataset of patients from a real ward.
  
  Lastly, the topic of \ac{AR}, especially in the scenario of \ac{AAL}, will be further investigated. The idea is to expand on the work of \cite{biagi2016stochastic,carnevali2015continuous} and on the investigation conducted in Jaén in order to enhance the \ac{AR} model. Among the possible enhancements there is the possibility of adding support for continuous sensors, such as a thermometer or an accelerometer, which is now lacking in the current model. Another important research activity for this topic is that of finding good datasets for \ac{AAL} \ac{AR} in order to apply techniques of process mining more accurately and refining the existing model, possibly following the recommendations shown in \cite{patara2015recommendations}.

\newpage
