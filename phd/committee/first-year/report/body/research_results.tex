% !TeX root = ../report-phd-first-year.tex
% !TeX encoding = UTF-8
% !TeX spellcheck = en_GB

\section*{Research and results}
  
  In this section, the main research conducted and the most relevant results will be shown. The main topic of the \acs{PhD} research is that of model-based quantitative analysis, especially in the scenario of partially observable systems.
  
  Before the start of the first year of the \ac{PhD}, a period of five months as a Research Fellow at University of Florence has been conducted, during which, under the supervision of Prof. Enrico Vicario, research activity has been started, following the same topics. In particular, during this period we produced the conference paper \cite{martina2016performance}, which focused on the performance evaluation of a mutual exclusion protocol (the Fischer's protocol) exploiting a technique for steady-state evaluation of \acp{MRP} \cite{logothetis1995markov}. Part of this work was also the implementation of the steady-state technique for \acp{MRP} described in \cite{logothetis1995markov} exploiting the \acp{API} of the Oris tool \cite{bucci2010oris}.
  
  The \ac{PhD} period started with the investigation and implementation of a technique for the transient analysis of \acp{MRP} under \textit{enabling restriction}, which characterises all those \acp{MRP} that have, at any given time, at most one \acr{GEN}
  
\newpage
