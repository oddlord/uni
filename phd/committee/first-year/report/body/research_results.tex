% !TeX root = ../report-phd-first-year.tex
% !TeX encoding = UTF-8
% !TeX spellcheck = en_GB

\section*{Research and results}
  
  In this section, the main research conducted and the most relevant results will be shown. The main topic of the \acr{PhD} research is that of model-based quantitative analysis, especially in the scenario of partially observable systems.
  
  Before the start of the first year of the \acr{PhD}, a period of five months as a Research Fellow at University of Florence has been conducted, during which, under the supervision of Prof. Enrico Vicario, research activity has been started, following the same topics. In particular, during this period we produced the conference paper \cite{martina2016performance}, which focused on the performance evaluation of a mutual exclusion protocol (the Fischer's protocol) exploiting a technique for steady-state evaluation of \acp{MRP} \cite{logothetis1995markov}. Part of this work was also the implementation of the steady-state technique for \acp{MRP} described in \cite{logothetis1995markov} exploiting the \acp{API} of the Oris tool \cite{bucci2010oris}.
  
  The \acr{PhD} period started with the investigation and implementation of a technique for the transient analysis of \acp{MRP} under \textit{enabling restriction}, which characterises all those \acp{MRP} that have, at any given time, at most one \acr{GEN}. In particular, the technique implemented has been studied from \cite{german1995transient} and implemented through the Oris tool \acp{API}.
  
  During the first year of the \acr{PhD} the \acr{LINFA} project has also been followed. The \acr{LINFA} project is a regional project funded by the Tuscany region that aims to develop a software for decision support for hospital staff members for drugs restocking. Drugs restocking can in fact be a hard and expensive process and by exploiting data processing and forecasting technique it can be made easier and cheaper. For this reason, techniques for model-based forecasting and decision support has been investigated. In particular, techniques that exploit \acp{MDP} \cite{bellman1957markovian} has been investigated and later implemented through the PRISM model checking tool \cite{KNP11} and a Java framework that generates an actualised PRISM model each time a drugs restock has to be issued, following the idea of \textit{models@runtime} \cite{blair2009models}.
  
  A compositional technique for transient analysis of \acp{MRP} has then been investigated. The idea was to combine both the technique for transient analysis of \acp{MRP} under enabling restriction shown in \cite{german1995transient} and the technique for transient analysis of \acp{MRP} under \textit{bounded regeneration}, which characterises all the \acp{MRP} that has no cycles between any two regenerations, shown in \cite{horvath2012transient}, exploiting non-deterministic analysis. In particular, this compositional technique would first perform non-deterministic analysis on an \ac{MRP} for each of its regenerations and classify them depending on which of the two conditions (enabling restriction and bounded regeneration) are satisfied. Depending on the result of this classification, the correct transient technique would then be applied to compute local and global kernels for that specific regeneration epoch, exploiting the fact that kernel rows, corresponding to different regenerative epochs, can be evaluated independently and thus with different techniques. When the whole local and global kernels have been computed, transient solution can be evaluated through the evaluation of Markov renewal equations. In order to evaluate those regenerations where none of the two conditions are satisfied, approximate evaluation has also been studied: the approximate technique investigated is based on the technique shown in \cite{horvath2012transient} and implements a guided transient analysis in order to explore first the ``most relevant'' transient classes. This compositional technique has been implemented exploiting the Oris tool \acp{API}. Results of this work, along with experimentation with the implemented technique, have been published in \cite{biagi2017exploiting}.
  
  Another investigation, regarding the analysis of assembly lines, has been pursued. Assembly lines analysis has become more and more important in these last years, in order to exploit techniques of data processing to maximise throughput and efficiency of the assembly lines, for example by dynamically adapting the production during runtime, according to the agenda of Industrie 4.0 \cite{hermann2016design}. The scenario investigated is that of an assembly line of sequential workstation, with transfer blocking and no buffering capacity, where each workstation can implement a complex workflow, with sequential, alternative of cyclic phases. At any given time, the assembly line under analysis could be inspected by an external observer, such as a human observer or a polling system, producing an observation where the status of each workstation and the specific phase of the producing ones could be observed. Such an observer, however, would only be able to partially observe the assembly line: firstly because some phases of the same workstation could be similar enough to produce the same observation (ambiguity on the logical state) but also because the external observer has no information regarding the time elapsed in the currently producing phases (ambiguity on the remaining times). In this context, we derived performance measure for the analysis of the assembly line: in particular we derived the \textit{\ac{TTD}}, representing the time until a certain workstation finishes working on a product, the \textit{\ac{TTI}}, representing the time until a certain workstation is available to accept a product from the previous one, and the \textit{\ac{TTSN}}, representing the time until a certain workstation start the production of a new product. Upper and lower bounds evaluation for the \acr{CDF} for the three measures was possible in a compositional fashion thanks to the positive correlation of remaining times in the producing phases and to the Key Renewal Theorem \cite{serfozo2009basics}. This work has been implemented through the Oris tool \acr{API}, thanks to which it was possible to validate the approach through a series of experiments that showed how the proposed approach results more feasible and more scalable than simulation. The detailed work, along with experimental results, have been published in \cite{biagi2017inspection}.
  
  Lastly, the field of \ac{AR} \cite{chen2011activity,patterson2003inferring,turaga2008machine} for \ac{AAL} \cite{cook2009ambient} and the field of dataset creation for \ac{AAL} \ac{AR} has been investigated. This investigation was part of an European secondment programme called REMIND and has been conducted while at the University of Jaén (\acused{UJA}\ac{UJA}), Spain, and it focused on exploring and comparing the different techniques for \ac{AAL} \ac{AR} developed by the \textit{\acr{STLab}}\footnote{\url{https://stlab.dinfo.unifi.it/}} in Florence and by the \textit{Sinbad\textsuperscript{2}}\footnote{\url{http://sinbad2.ujaen.es/}} research group in Jaén. In particular, techniques for \ac{AAL} \ac{AR} based on stochastic models \cite{biagi2016stochastic,carnevali2015continuous} and on fuzzy logic \cite{medina2015activity}, as well as several works on the creation of datasets for \ac{AAL} \ac{AR} \cite{patara2015recommendations,quesada2015generation}, have been studied and compared and joint proposals have been suggested in order to take advantage of the different techniques studied.
  
\newpage
