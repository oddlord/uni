% !TeX root = ../report-phd-first-year.tex
% !TeX encoding = UTF-8
% !TeX spellcheck = en_GB

\section*{Research and results}
  
  In this section, the main research conducted and the most relevant results will be shown. The main topic of the \acr{PhD} research is that of model-based quantitative analysis, especially in the scenario of partially observable systems.
  
  Before the start of the first year of the \acr{PhD}, a period of five months as a Research Fellow at University of Florence has been conducted, during which, under the supervision of Prof. Enrico Vicario, research activity has been started, following the same topics. In particular, during this period we produced the conference paper \cite{martina2016performance}, which focused on the performance evaluation of a mutual exclusion protocol (the Fischer's protocol) exploiting a technique for steady-state evaluation of \acp{MRP} \cite{logothetis1995markov}. Part of this work was also the implementation of the steady-state technique for \acp{MRP} described in \cite{logothetis1995markov} exploiting the \acp{API} of the Oris tool \cite{bucci2010oris}.
  
  The \acr{PhD} period started with the investigation and implementation of a technique for the transient analysis of \acp{MRP} under \textit{enabling restriction}, which characterises all those \acp{MRP} that have, at any given time, at most one \acr{GEN}. In particular, the technique implemented has been studied from \cite{german1995transient} and implemented through the Oris tool \acp{API}.
  
  During the first year of the \acr{PhD} the \acr{LINFA} project has also been followed. The \acr{LINFA} project is a regional project funded by the Tuscany region that aims to develop a software for decision support for hospital staff members for drugs restocking. Drugs restocking can in fact be a hard and expensive process and by exploiting data processing and forecasting technique it can be made easier and cheaper. For this reason, techniques for model-based forecasting and decision support has been investigated. In particular, techniques that exploit \acp{MDP} \cite{bellman1957markovian} has been investigated and later implemented through the PRISM model checking tool \cite{KNP11} and a Java framework that generates an actualised PRISM model each time a drugs restock has to be issued, following the idea of \textit{models@runtime} \cite{blair2009models}.
  
  A compositional technique for transient analysis of \acp{MRP} has then been investigated. The idea was to combine both the technique for transient analysis of \acp{MRP} under enabling restriction shown in \cite{german1995transient} and the technique for transient analysis of \acp{MRP} under \textit{bounded regeneration}, which characterises all the \acp{MRP} that has no cycles between any two regenerations, shown in \cite{horvath2012transient}, exploiting non-deterministic analysis. In particular, this compositional technique would first perform non-deterministic analysis on an \ac{MRP} for each of its regenerations and classify them depending on which of the two conditions (enabling restriction and bounded regeneration) are satisfied. Depending on the result of this classification, the correct transient technique would then be applied to compute local and global kernels for that specific regeneration epoch, exploiting the fact that kernel rows, corresponding to different regenerative epochs, can be evaluated independently and thus with different techniques. When the whole local and global kernels have been computed, transient solution can be evaluated through the evaluation of Markov renewal equations. In order to evaluate those regenerations where none of the two conditions are satisfied, approximate evaluation has also been studied: the approximate technique investigated is based on the technique shown in \cite{horvath2012transient} and implements a guided transient analysis in order to explore first the ``most relevant'' transient classes. Results of this work have been published in \cite{biagi2017exploiting}.
  
\newpage
