% !TeX root = ../phd-1st-year-presentation.tex
% !TeX encoding = UTF-8
% !TeX spellcheck = en_GB

\section{A hybrid technique for MRP transient analysis}
  \begin{frame}{A hybrid technique for MRP transient analysis}
    Transient analysis of Markov Regenerative Processes (MRP)\\
    employing different techniques for different regenerative epochs
    
    \vspace{1em}
    The basics:
    \begin{itemize}
      \item Exact techniques require specific conditions to be met
      \begin{itemize}
        \item different techniques require different conditions
      \end{itemize}
      \item Kernel rows of different epochs can be evaluated independently
    \end{itemize}
    
    \vspace{1em}
    The idea:
    \begin{itemize}
      \item Evaluate each kernel row with a different technique
      \begin{itemize}
        \item corresponding to the condition enabled in that epoch
        \item eventually with an approximate technique, if no conditions are met
      \end{itemize}
      \item Compute transient probabilities with Markov Renewal Equations
    \end{itemize}
  \end{frame}
  
  \subsection{Techniques for MRP transient analysis}
    \begin{frame}{Techniques for MRP transient analysis}
      \begin{minipage}{0.6\textwidth}
        \textbf{Analysis under enabling restriction}\footnotemark
        \begin{itemize}
          \item at most one GEN enabled in each state
        \end{itemize}
      \end{minipage}
      \begin{minipage}{0.35\textwidth}
        \begin{center}\scalebox{0.45}{\begin{tikzpicture}[
    node distance=1.6cm,
    label distance=0.05mm,
    >=stealth',
    bend angle=45,
    auto,
    font=\sffamily,
    font=\small
  ]
  
  \def\timedWidth{2mm}
  \def\immWidth{1mm}
  \def\transitionHeight{6mm}
  
  \tikzstyle{inhib}=[-o]
  \tikzstyle{imm}=[
    rectangle,
    draw=black!100,
    fill=black!100,
    minimum height=\transitionHeight,
    minimum width=\immWidth,
    inner xsep=0mm
  ]
  
  \tikzstyle{gen}=[
    rectangle,
    thick,
    draw=black!100,
    fill=black!80,
    minimum height=\transitionHeight,
    minimum width=\timedWidth
  ]
  
  \tikzstyle{det}=[
    rectangle,
    thick,
    draw=black!75,
    fill=black!35,
    minimum height=\transitionHeight,
    minimum width=\timedWidth
  ]
  
  \tikzstyle{exp}=[
    rectangle,
    thick,
    draw=black!75,
    fill=white!20,
    minimum height=\transitionHeight,
    minimum width=\timedWidth
  ]
  
  \tikzstyle{hor}=[
    minimum height=\timedWidth,
    minimum width=\transitionHeight
  ]
  
  \tikzstyle{immhor}=[
    minimum height=\immWidth,
    minimum width=\transitionHeight
  ]
  
  \tikzstyle{place}=[
    circle,
    thick,
    minimum size=6mm,
    draw=black
  ]
  
  \def\upperplaceheight{0}
  \def\middleplaceheight{-1}
  \def\lowerplaceheight{-2}
  	
  \begin{scope}[
    decorate,
    scale=1.6,
    decoration={
    	random steps,
    	segment length=0.5mm,
    	amplitude=0.15pt
    },
    token distance=0.75ex
  ]
    
    \node [place,tokens=1] at (0,0) (G1) [label={[align=center]above:$\pl{G_1}$}] {};	
    \node [place,tokens=0] at (1.5,0) (Restart) [label={[align=center]above:$\pl{Restart}$}] {};
    \node [place,tokens=1] at (3.5,0) (E1) [label={[align=center]above:$E_1$}] {};
    \node [place,tokens=0] at (5,-1.2) (E2) [label={[align=center]right:$E_2$}] {};
    
    \node [gen] at (0.75,0) (gen1) [label={[align=center]below:$\tr{gen_1}$\\\textsc{uni}[2,4]\\$E_1 \leftarrow 0$\\$E_2 \leftarrow 0$}] {} edge [pre]  (G1) edge [post] (Restart);
    
    \node [det] at (2.25,0) (restart) [label={[align=center]above:$\tr{restart}$\\\textsc{det}(1)}] {} edge [pre] (Restart) edge [post] (E1);
    \draw [post] (restart) -- ++(0,-1.2) -- (0,-1.2) -- (G1);
    
    \node [exp] at (4.25,0) (exp1) [label={[align=center]below:$\tr{exp_1}$\\\textsc{EXP}(1)}] {} edge [pre] (E1);
    \draw [post] (exp1) -- ++(0.75,0) -- (E2);
    
    \node [exp] at (4.25,-1.2) (exp2) [label={[align=center]below:$\tr{exp_2}$\\\textsc{EXP}(1)}] {} edge [pre] (E2);
    \draw [post] (exp2) -- ++(-0.75,0) -- (E1);
  
  \end{scope}
\end{tikzpicture}
 
}\end{center}
      \end{minipage}
      
      \begin{minipage}{0.6\textwidth}
        \textbf{Analysis with stochastic state classes}\footnotemark
        \begin{itemize}
          \item a regeneration is always reached\\
            within a bounded number of events
          \begin{itemize}
            \item i.e. no cycles without regenerations
          \end{itemize}
          \item a.k.a. bounded regeneration
        \end{itemize}
      \end{minipage}
      \begin{minipage}{0.35\textwidth}
        \begin{center}\scalebox{0.45}{\input{img/pn_bounded_regeneration}}\end{center}
      \end{minipage}
      
      \begin{minipage}{0.6\textwidth}
        \textbf{Approximate analysis}
        \begin{itemize}
          \item usable when no conditions are met
        \end{itemize}
      \end{minipage}
      \begin{minipage}{0.35\textwidth}
        \begin{center}\scalebox{0.45}{\begin{tikzpicture}[
    node distance=1.6cm,
    label distance=0.05mm,
    >=stealth',
    bend angle=45,
    auto,
    font=\sffamily,
    font=\small
  ]
  
  \def\timedWidth{2mm}
  \def\immWidth{1mm}
  \def\transitionHeight{6mm}
  
  \tikzstyle{inhib}=[-o]
  \tikzstyle{imm}=[
    rectangle,
    draw=black!100,
    fill=black!100,
    minimum height=\transitionHeight,
    minimum width=\immWidth,
    inner xsep=0mm
  ]
  
  \tikzstyle{gen}=[
    rectangle,
    thick,
    draw=black!100,
    fill=black!80,
    minimum height=\transitionHeight,
    minimum width=\timedWidth
  ]
  
  \tikzstyle{det}=[
    rectangle,
    thick,
    draw=black!75,
    fill=black!35,
    minimum height=\transitionHeight,
    minimum width=\timedWidth
  ]
  
  \tikzstyle{exp}=[
    rectangle,
    thick,
    draw=black!75,
    fill=white!20,
    minimum height=\transitionHeight,
    minimum width=\timedWidth
  ]
  
  \tikzstyle{hor}=[
    minimum height=\timedWidth,
    minimum width=\transitionHeight
  ]
  
  \tikzstyle{immhor}=[
    minimum height=\immWidth,
    minimum width=\transitionHeight
  ]
  
  \tikzstyle{place}=[
    circle,
    thick,
    minimum size=6mm,
    draw=black
  ]
  
  \def\upperplaceheight{0}
  \def\middleplaceheight{-1}
  \def\lowerplaceheight{-2}
  	
  \begin{scope}[
    decorate,
    scale=1.6,
    decoration={
    	random steps,
    	segment length=0.5mm,
    	amplitude=0.15pt
    },
    token distance=0.75ex
  ]
    
    \node [place,tokens=1] at (0,0) (G1) [label={[align=center]above:$\pl{G_1}$}] {};	
    \node [place,tokens=0] at (1.5,0) (Restart) [label={[align=center]above:$\pl{Restart}$}] {};
    \node [place,tokens=1] at (3.5,0) (G2) [label={[align=center]above:$G_2$}] {};
    \node [place,tokens=0] at (5,-1.2) (G3) [label={[align=center]right:$G_3$}] {};
    
    \node [gen] at (0.75,0) (gen1) [label={[align=center]below:$\tr{gen_1}$\\\textsc{uni}[2,4]\\$G_2 \leftarrow 0$\\$G_3 \leftarrow 0$}] {} edge [pre]  (G1) edge [post] (Restart);
    
    \node [det] at (2.25,0) (restart) [label={[align=center]above:$\tr{restart}$\\\textsc{det}(1)}] {} edge [pre] (Restart) edge [post] (G2);
    \draw [post] (restart) -- ++(0,-1.2) -- (0,-1.2) -- (G1);
    
    \node [gen] at (4.25,0) (gen2) [label={[align=center]below:$\tr{gen_2}$\\\textsc{ERLANG}(2,1)}] {} edge [pre] (G2);
    \draw [post] (gen2) -- ++(0.75,0) -- (G3);
    
    \node [gen] at (4.25,-1.2) (gen3) [label={[align=center]below:$\tr{gen_3}$\\\textsc{ERLANG}(2,1)}] {} edge [pre] (G3);
    \draw [post] (gen3) -- ++(-0.75,0) -- (G2);
  
  \end{scope}
\end{tikzpicture}
 
}\end{center}
      \end{minipage}
      
      \addtocounter{footnote}{-1}
      \footnotetext{German, R., Logothetis, D., \& Trivedi, K. S. (1995, October). Transient analysis of Markov regenerative stochastic Petri nets: A comparison of approaches. In Petri Nets and Performance Models, 1995., Proceedings of the Sixth International Workshop on (pp. 103-112). IEEE.}
      \stepcounter{footnote}
      \footnotetext{Horváth, A., Paolieri, M., Ridi, L., \& Vicario, E. (2012). Transient analysis of non-Markovian models using stochastic state classes. Performance Evaluation, 69(7), 315-335.}
    \end{frame}
  
  \subsection{Classification of epochs}
    \begin{frame}{Classification of epochs}
      Through \textbf{non-deterministic analysis}
      \begin{itemize}
        \item State Class Graphs (SCG) are built
        \item for each regenerative epoch
      \end{itemize}
      
      \vspace{2em}
      By visiting each SCG, epochs are classified
      \begin{itemize}
        \item enabling restriction
        \begin{itemize}
          \item if at most one GEN is enabled in any state
        \end{itemize}
        \item bounded regeneration
        \begin{itemize}
          \item if no cycle is present
        \end{itemize}
      \end{itemize}
    \end{frame}
    
  \subsection{Iterative approximate technique}
    \begin{frame}{Iterative approximate technique}
      Based on analysis with stochastic state classes
      \begin{itemize}
        \item truncated after enough precision is met
      \end{itemize}
      
      \vspace{2em}
      Improvement with heuristics
      \begin{enumerate}
        \item expand at most $\nu_{start}$ nodes for non restricted epochs
        \item identify the truncated node $\Phi$ with highest reaching probability
        \begin{itemize}
          \item based on steady-state analysis of the embedded DTMC
        \end{itemize}
        \item expand at most $\nu_{iter}$ nodes from $\Phi$
        \item if at least $\nu_{max}$ nodes expanded, stop
        \begin{itemize}
          \item otherwise, return to step 2
        \end{itemize}
      \end{enumerate}
    \end{frame}
