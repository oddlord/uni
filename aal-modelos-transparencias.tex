% !TeX program = pdflatex
% !TeX encoding = UTF-8
% !TeX spellcheck = es_ES

\documentclass[9pt, handout]{beamer}

% !TeX root = ../aal-modelos-transparencias.tex
% !TeX encoding = UTF-8
% !TeX spellcheck = es_ES

%********************************************************************
% Packages
%********************************************************************

\usepackage[spanish]{babel}
\usepackage[utf8]{inputenc}
\usepackage[T1]{fontenc}

\usepackage{amsmath}
\usepackage{amsfonts}
\usepackage{amssymb}
\usepackage{calligra}
\usepackage{graphicx}
\usepackage{tikz}
\usetikzlibrary{arrows,automata,shapes,calc,backgrounds,positioning}
\usepackage{times}

\graphicspath{{img/}}

%********************************************************************
% Beamer styles
%********************************************************************

\usetheme{Montpellier}
\usecolortheme{default}
\setbeamercovered{dynamic}
\newcommand{\nologo}{\setbeamertemplate{logo}{}} % command to set the logo to nothing

\setbeamertemplate{navigation symbols}{}
\setbeamertemplate{footline}[frame number]

\setbeamertemplate{footline}[frame number]
\setbeamertemplate{navigation symbols}{}

\usebackgroundtemplate{
  \begin{tikzpicture}
    \node[opacity=0.05] {\includegraphics[]{logoUnifi.png}};
  \end{tikzpicture}
}

\AtBeginSection[]{
  \begin{frame}
  \vfill
  \centering
  \begin{beamercolorbox}[sep=8pt,center,shadow=true,rounded=true]{title}
    \usebeamerfont{title}\insertsectionhead\par%
  \end{beamercolorbox}
  \vfill
  \end{frame}
}


\title[AAL y modelos aleatorios]{Ambient Assisted Living y modelos aleatorios}
\author{\textbf{Tommaso Papini}}
\institute{
  STLab, Departamiento de la Ingenieria de la Informacíon, Universidad de Florencia, Italia,\\
  {tommaso.papini@unifi.it}
}
\date{
  29 de Mayo 2017\\
  {\small Departamento de Informática, Universidad de Jaén, España}
}

\begin{document}

  \begin{frame}
    \titlepage
    \begin{itemize}
      \item Ambient Assisted Living
      \item Análisis de entornos inteligentes
    \end{itemize}
  \end{frame}

  \begin{frame}{Overview}
    %\tiny
    \tableofcontents
  \end{frame}
  
  \section{Ambient Assisted Living}
    
    \begin{frame}{Ambient Assisted Living}
      El \textbf{Ambient Assisted Living} es un sector de investigación que tiene como objetivo lo de ayudar las personas que viven en \textit{entornos inteligentes} (es decir, dotado de sensores y actuadores) explotando a la tecnología de sensores y de procesamiento de datos.
    \end{frame}
    
    \subsection{Objetivos}
      \begin{frame}{Objetivos}
        \pause
        Un \textit{entorno inteligente} es un sistema \textbf{parcialmente observable}:
        \pause
        \begin{itemize}
          \item el estado efectivo del sistema resulta escondido
          \pause
          \item solo se pueden observar eventos (\textit{observaciones}) emitidos por el sistema (por ej. la activación de un sensor)\\[1em]
        \end{itemize}
        
        \pause
        
        Principales análisis de interese:
        \pause
        \begin{itemize}
          \item \textbf{Diagnosis}: estimar cual es el estado efectivo actual del sistema a partir de las observaciones registradas
          \pause
          \item \textbf{Predicción}: estimar cual será el estado efectivo del sistema después de una determinada cantidad de tiempo
          \item \textbf{Planificación de acciones}: elegir la acción optima y entre cuanto tiempo ir actuarla para evitar situaciones críticas\\[1em]
        \end{itemize}
        
        \pause
        
        Análisis en linea:
        \pause
        \begin{itemize}
          \item la análisis en linea intenta analizar a un entorno inteligente \textit{mientras} está evolucionando
        \end{itemize}
      \end{frame}
    
  \section{Análisis de entornos inteligentes}
  
    \begin{frame}{Modelos estadísticos}
      Los \textit{modelos estadísticos} representan una aproximación de sistemas donde se modela:
      \begin{itemize}
        \item la evolución del estado del sistema
        \item parámetros estadísticos que definen como el sistema pasa de un estado al otro
      \end{itemize}
      \begin{center}
        \colorbox{white}{\includegraphics[scale=0.28]{708px-HMMGraph.png}}
      \end{center}
    \end{frame}
    
    \begin{frame}{Datasets anotados}
      Un \textit{dataset anotado} es un dataset donde hay:
      \begin{itemize}
        \item los eventos registrados y cuando han pasado (marca temporal)
        \item anotaciones manuales de la evolución del estado efectivo del sistema (con intervalos temporales por cada estado)
      \end{itemize}
      Un ejemplo clásico de dataset anotado para AAL es el dataset de \textit{van Kasteren}\footnote{\url{https://sites.google.com/site/tim0306/datasets}}\footnote{Van Kasteren, T., Noulas, A., Englebienne, G. and Kröse, B., 2008, September. Accurate activity recognition in a home setting. In Proceedings of the 10th international conference on Ubiquitous computing (pp. 1-9). ACM.}\\[1em]
      
      \includegraphics[scale=0.8]{activity_event_formulation.pdf}
    \end{frame}
    
    \begin{frame}{Process mining: de datasets anotados a modelos estadísticos}
      Con el término \textbf{process mining} se indica un conjunto de técnicas para construir un modelo estadístico de un sistema parcialmente observable a partir de un dataset anotado de este mismo sistema.\\[1em]
      
      El process mining está compuesto por dos técnicas principales:
      \begin{itemize}
        \item \textbf{Process elicitation}: construye un modelo discreto (es decir, sin informaciones sobre la permanencia en los estados del sistema) a partir de los eventos y actividades anotados en el dataset
        \item \textbf{Process ehnancement}: añade una visión temporal continua a un modelo discreto introduciendo parámetros estadísticos que describen como el sistema evoluciona a lo largo del tiempo utilizando medidas estadísticas sacadas por el dataset
      \end{itemize}
    \end{frame}
  
    \begin{frame}{Diagnosis, predicción y planificación de acciones}
      \begin{center}
        \includegraphics[scale=0.34]{architecture.png}
      \end{center}
    \end{frame}
  
    \subsection{Diagnosis}
      \begin{frame}{Diagnosis: tiempo discreto}
        Los \textbf{Modelos Ocultos de Márkov} (Hidden Markov Model, HMM):
        \begin{itemize}
          \item modelan a un sistema parcialmente observable sin tener en cuenta del tiempo de permanencia en cada estado
          \item asocian a cada estado efectivo del sistema una distribución discreta sobre los eventos observables
          \item modelan a un sistema donde los estados efectivos evolucionan como una \textit{Cadena de Márkov Tiempo Discreto} (Discrete Time Markov Chain, DTMC)
          
        \end{itemize}
        \begin{center}
          \includegraphics[scale=0.25]{HMM.png}
        \end{center}
        Diagnosis con HMM se puede lograr con algoritmos clásicos como el \textit{algoritmo de Viterbi}.
      \end{frame}
      
    \subsection{Predicción}
      \begin{frame}{TODO!!!}
      \end{frame}
      
    \subsection{Planificación de datos}
      \begin{frame}{TODO!!!}
      \end{frame}

\end{document}
