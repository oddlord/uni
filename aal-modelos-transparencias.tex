% !TeX program = pdflatex
% !TeX encoding = UTF-8
% !TeX spellcheck = es_ES

\documentclass[9pt]{beamer}

% !TeX root = ../aal-modelos-transparencias.tex
% !TeX encoding = UTF-8
% !TeX spellcheck = es_ES

%********************************************************************
% Packages
%********************************************************************

\usepackage[spanish]{babel}
\usepackage[utf8]{inputenc}
\usepackage[T1]{fontenc}

\usepackage{amsmath}
\usepackage{amsfonts}
\usepackage{amssymb}
\usepackage{calligra}
\usepackage{graphicx}
\usepackage{tikz}
\usetikzlibrary{arrows,automata,shapes,calc,backgrounds,positioning}
\usepackage{times}

\graphicspath{{img/}}

%********************************************************************
% Beamer styles
%********************************************************************

\usetheme{Montpellier}
\usecolortheme{default}
\setbeamercovered{dynamic}
\newcommand{\nologo}{\setbeamertemplate{logo}{}} % command to set the logo to nothing

\setbeamertemplate{navigation symbols}{}
\setbeamertemplate{footline}[frame number]

\setbeamertemplate{footline}[frame number]
\setbeamertemplate{navigation symbols}{}

\usebackgroundtemplate{
  \begin{tikzpicture}
    \node[opacity=0.05] {\includegraphics[]{logoUnifi.png}};
  \end{tikzpicture}
}

\AtBeginSection[]{
  \begin{frame}
  \vfill
  \centering
  \begin{beamercolorbox}[sep=8pt,center,shadow=true,rounded=true]{title}
    \usebeamerfont{title}\insertsectionhead\par%
  \end{beamercolorbox}
  \vfill
  \end{frame}
}


\title[AAL basada en modelos]{Ambient Assisted Living basada en modelos}
\author{\textbf{Tommaso Papini}}
\institute{
  STLab, Departamiento de la Ingenieria de la Informacíon, Universidad de Florencia, Italia,\\
  {tommaso.papini@unifi.it}
}
\date{
  29 de Mayo 2017\\
  {\small Departamento de Informática, Universidad de Jaén, España}
}

\begin{document}

  \begin{frame}
    \titlepage
    \begin{itemize}
      \item Análisis cuantitativa basada en modelos
      \item Reconocimiento de Actividades
%      \begin{itemize}
%        \item diagnosis
%        \item predicción
%        \item planificación de acciones
%      \end{itemize}
      \item Datasets para AAL
    \end{itemize}
  \end{frame}

  \begin{frame}{Overview}
    %\tiny
    \tableofcontents
  \end{frame}
  
  \section{Análisis cuantitativa basada en modelos}
    
    \subsection{Redes de Petri}
      \begin{frame}{TODO!!!}
      \end{frame}
      
    \subsection{Análisis de transición}
      \begin{frame}{TODO!!!}
      \end{frame}

  \section{Reconocimiento de Actividades}
  
    \begin{frame}{Dos artículos principales}
      \pause
      \begin{itemize}
        \item Carnevali, L., Nugent, C., Patara, F. and Vicario, E., 2015, September. \textbf{A continuous-time model-based approach to activity recognition for ambient assisted living}. In International Conference on Quantitative Evaluation of Systems (pp. 38-53). Springer International Publishing.
        \pause
        \item Biagi, M., Carnevali, L., Paolieri, M., Patara, F. and Vicario, E., 2016, October. \textbf{A Stochastic Model-Based Approach to Online Event Prediction and Response Scheduling}. In European Workshop on Performance Engineering (pp. 32-47). Springer International Publishing.
      \end{itemize}
    \end{frame}
    
    \subsection{Diagnosis y predicción}
      \begin{frame}{Diagnosis y predicción}
        \pause
        Un \textit{entorno inteligente} (es decir, dotado de sensores y actuadores) es un sistema \textbf{parcialmente observable}:
        \pause
        \begin{itemize}
          \item el estado efectivo del sistema resulta escondido
          \pause
          \item solo se pueden observar eventos (\textit{observaciones}) emitidos por el sistema (por ej. la activación de un sensor)\\[1.5em]
        \end{itemize}
        
        \pause
        
        Reconocimiento de Actividades:
        \pause
        \begin{itemize}
          \item \textbf{Diagnosis}: estimar cual es el estado efectivo actual del sistema a partir de las observaciones registradas
          \pause
          \item \textbf{Predicción}: estimar cual será el estado efectivo del sistema después de una determinada cantidad de tiempo a partir del resultado de la diagnosis\\[1.5em]
        \end{itemize}
        
        \pause
        
        El Reconocimiento de Actividades es fundamental en el desarrollo de sistemas inteligentes en cuanto actúa de puente entre los datos recibidos por los sensores y la semántica de alto nivel de las aplicaciones.
        
        % TODO: figura qui???
      \end{frame}
      
      \begin{frame}{HMM y CRF}
        \pause
        Los \textit{Modelos Ocultos de Márkov} (HMM, Hidden Markov Model) y los \textit{Campos Aleatorios Condicionales} (CRF, Conditional Random Field) ofrecen soluciones clásicas para realizar diagnosis y predicción en sistemas parcialmente observables en tiempo \textbf{discreto}.\\[1.5em]
        
        \pause
        
        El objetivo principal de las técnicas basadas en modelos para sistemas parcialmente observables es realizar la misma diagnosis y predicción, pero en tiempo \textbf{continuo}.
      \end{frame}
      
      \begin{frame}{Diagnosis y predicción con modelos}
        \pause
        La idea es explotar las técnicas de análisis cuantitativa basada en modelos para realizar diagnosis y predicción.\\[1em]
        
        \pause
        
        Pasos principales:
        \pause
        \begin{enumerate}
          \item Obtener a un dataset de observaciones \textit{anotado} (es decir, con observaciones y actividades efectivas en un intervalo de tiempo)\footnote{Van Kasteren, T., Noulas, A., Englebienne, G. and Kröse, B., 2008, September. \textbf{Accurate activity recognition in a home setting}. In Proceedings of the 10th international conference on Ubiquitous computing (pp. 1-9). ACM.}
          \pause
          \item Calcular medidas estadísticas
          \begin{itemize}
            \item correlación entre eventos y actividades
            \item duración de actividades
            \item inter-tiempo entre eventos durante actividades
          \end{itemize}
          \pause
          \item Construir a un modelo del sistema
          \begin{itemize}
            \item process elicitation
            \item process enhancement
          \end{itemize}
          \pause
          \item Análisis de transición
        \end{enumerate}
      \end{frame}
        
    \subsection{Planificación de acciones}
      \begin{frame}{TODO!!!}
      \end{frame}
            
  \section{Datasets para AAL}
    
    \subsection{TODO!!!}
      \begin{frame}{TODO!!!}
      \end{frame}

\end{document}
