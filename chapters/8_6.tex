\section*{Esercizio 8.6}

    \enunciato{Si calcoli formalmente, usando la semantica denotazionale di TINY, l'insieme degli stati a partire dai quali eseguendo il comando seguente si ha divergenza:
        \begin{align*}
            \mbox{\textbf{while}}\;& x>0\; \mbox{\textbf{do}}\\
            & \mbox{\textbf{if}}\; x<3\; \mbox{\textbf{then}}\; x:=x-1\; \mbox{\textbf{else noaction}}.
        \end{align*}
    }
    
%    A seconda del valore assegnato dalla memoria all'identificatore di variabile $x$ si avrà la divergenza o meno del comando proposto. Si nota inoltre che, non comparendo mai l'espressione \textbf{read} o il comando \textbf{output} $e$, i valori di input e di output dello stato saranno indifferenti per lo studio della divergenza.\\
%    Sia quindi $C$ il comando proposto e sia $\sigma = <i,o,m>$ lo stato in esame, con $mx=v$, procediamo allora per casi, distinguendo in base ai possibili valori assegnati ad $x$ nella memoria:
%    \begin{itemize}
%        \item $v = unbound$\\
%            Banalmente, se ad $x$ è assegnato il valore \textit{unbound}, significa che $x$ non è legato ad alcun valore in memoria ed il risultato è la situazione di errore.\\
%            Formalmente, la sola espressione $x$ viene interpretata come segue:
%            $$\mathcal{E}\doublebracket{ x } \sigma = error,$$
%            per definizione di $\mathcal{E}\doublebracket{ x }$ e per il fatto che $mx=unbound$ in $\sigma$. Quindi l'interpretazione della condizione $x>0$, essendo un'operazione tra naturali, risulta essere:
%            $$\mathcal{E}\doublebracket{ x>0 } \sigma = error,$$
%            dovuto al fatto che se l'interpretazione del primo argomento risulta dar luogo a un errore, allora viene propagato l'errore, secondo la definizione di $\mathcal{E}\doublebracket{ e_1\; nop\; n_2 }$.
%    \end{itemize}
    
    Per semplicità, analizziamo le interpretazioni semantiche di espressioni e comandi che formano il comando while proposto prima di passare alla prova vera e propria. Inoltre considereremo tutte le interpretazioni semantiche depurate dai controlli d'errore: non è difficile infatti vedere che ogni volta che si verifica un errore il sistema di interpretazione semantica delle dispense non fa altro che propagare l'errore, garantendo quindi convergenza. Infine si assume che lo stato $\sigma_k$ sia equivalente allo stato in forma estesa $<i_k, o_k, m_k>$, quindi ad esempio se scriviamo $m_2$ sappiamo che questo indica la funzione memoria dello stato $\sigma_2$, senza bisogno di appesantire la dimostrazione.\\
    \\
    Tutte le interpretazioni verranno calcolate già in un generico stato $\sigma$, per semplicità di scrittura.\\
    \\
    Iniziamo col definire l'interpretazione delle tre costanti naturali presenti nel comando, ovvero $0$, $1$ e $3$:
    
    \begin{equation*}
        \mathbb{E}\doublebracket{0}\sigma = <0, \sigma>
    \end{equation*}
    \begin{equation*}
        \mathbb{E}\doublebracket{1}\sigma = <1, \sigma>
    \end{equation*}
    \begin{equation*}
        \mathbb{E}\doublebracket{3}\sigma = <3, \sigma>
    \end{equation*}
    
    Sfruttando la definizione di interpretazione di variabile (depurata da controlli d'errore) e la notazione assunta sugli stati, abbiamo la seguente interpretazione per $x$ in uno stato $\sigma$:
    
    \begin{equation*}
        \mathbb{E}\doublebracket{x}\sigma = <mx, \sigma>
    \end{equation*}
    
    Vediamo adesso l'interpretazione di $x>0$, sostituendo opportunamente le interpretazioni già calcolate:
    
    \begin{align*}
        \mathbb{E}\doublebracket{x>0}\sigma =\;& \mbox{let} <v_1,\sigma_1> \mbox{be}\; \mathbb{E}\doublebracket{x}\sigma\; \mbox{in}\;\\
        &\mbox{let} <v_2,\sigma_2> \mbox{be}\; \mathbb{E}\doublebracket{0}\sigma_1\; \mbox{in}\;\\
        &<v_1\; \mbox{gt}\; v_2, \sigma_2>\\
        =\;& <<mx,\sigma>\; \mbox{gt}\; <0, \sigma>, \sigma>
    \end{align*}
    
    Il simbolo $gt$ (greater than) è il simbolo semantico corrispondente al simbolo sintattico $-$, nonnché un'istanza di $nop$. Si noti come, grazie al fatto che nessuna delle due valutazioni modifica lo stato, possiamo utilizzare lo stesso $\sigma$, alleggerendo così la notazione.\\
    \\
    Procediamo in modo analogo per $x-1$ e per $x<3$, altre due operazioni tra naturali:
    
    \begin{align*}
        \mathbb{E}\doublebracket{x-1}\sigma =\;& \mbox{let} <v_1,\sigma_1> \mbox{be}\; \mathbb{E}\doublebracket{x}\sigma\; \mbox{in}\;\\
        &\mbox{let} <v_2,\sigma_2> \mbox{be}\; \mathbb{E}\doublebracket{1}\sigma_1\; \mbox{in}\;\\
        &<v_1\; \mbox{minus}\; v_2, \sigma_2>\\
        =\;& <<mx,\sigma>\; \mbox{minus}\; <1, \sigma>, \sigma>
    \end{align*}
    
    \begin{align*}
        \mathbb{E}\doublebracket{x<3}\sigma =\;& \mbox{let} <v_1,\sigma_1> \mbox{be}\; \mathbb{E}\doublebracket{x}\sigma\; \mbox{in}\;\\
        &\mbox{let} <v_2,\sigma_2> \mbox{be}\; \mathbb{E}\doublebracket{3}\sigma_1\; \mbox{in}\;\\
        &<v_1\; \mbox{lt}\; v_2, \sigma_2>\\
        =\;& <<mx,\sigma>\; \mbox{lt}\; <3, \sigma>, \sigma>
    \end{align*}
    
    Permettendoci un piccolo abuso di notazione, definiamo le tre notazioni seguenti, relative a due valori naturali e ad uno stato, delle operazioni ``maggiore di'', ``meno'' e ``minore di'', che intuitivamente rappresentano il solo valore ottenuto dall'esecuzione di quelle operazioni su quei valori e su quello stato:
    
    \begin{align*}
        GT_{i,j,\sigma} &\equiv <i,\sigma>\; \mbox{gt}\; <j, \sigma>\\
        M_{i,j,\sigma} &\equiv <i,\sigma>\; \mbox{minus}\; <j, \sigma>\\
        LT_{i,j,\sigma} &\equiv <i,\sigma>\; \mbox{lt}\; <j, \sigma>
    \end{align*}
    
    Inoltre useremo la notazione, in forma compatta, $\sigma[v/x] \equiv <i,o,m[v/x]>$.\\
    \\
    Calcoliamo adesso l'interpretazione del comando di assegnazione $x:=x-1$ nello stato $\sigma$:
    
    \begin{align*}
        \mathbb{C}\doublebracket{x:=x-1}\sigma =\;& \mbox{let} <v_1,\sigma_1> \mbox{be}\; \mathbb{E}\doublebracket{x-1}\; \mbox{in}\\
        &<i_1,o_1,m_1[v_1/x]>\\
        =\;& <i,o,m[M_{mx,1,\sigma}/x]>\\
        =\;& \sigma[M_{mx,1,\sigma}/x]
    \end{align*}
    
    Si può facilmente dimostrare che:
    
    \begin{align*}
        GT_{mx,j,\sigma[M_{mx,h,\sigma}/x]} &= GT_{mx, (j+h), \sigma}\\
        M_{mx,j,\sigma[M_{mx,h,\sigma}/x]} &= M_{mx, (j+h), \sigma}\\
        LT_{mx,j,\sigma[M_{mx,h,\sigma}/x]} &= LT_{mx, (j+h), \sigma}\\
        \sigma[M_{mx,a,\sigma}/x][M_{mx,b,\sigma[M_{mx,a,\sigma}/x]}/x] &= \sigma[M_{mx,(a+b),\sigma}/x]
    \end{align*}
    
    Inoltre risulta intuitivo che:
    
    \begin{align*}
        GT_{a,b,\sigma} \rightarrow (GT_{a,c,\sigma} \rightarrow \alpha, \gamma), \beta &= GT_{a,b,\sigma} \rightarrow \alpha, \beta, &&\mbox{con $b\geq c$}\\
        LT_{a,b,\sigma} \rightarrow (GT_{a,c,\sigma} \rightarrow \alpha, \gamma), \beta &= LT_{a,b,\sigma} \rightarrow \gamma, \beta, &&\mbox{con $b > c+1$}\\
        LT_{a,b,\sigma} \rightarrow \alpha, (GT_{a,c,\sigma} \rightarrow \beta, \gamma) &= LT_{a,b,\sigma} \rightarrow \alpha, \beta, &&\mbox{con $b > c$}\\
        LT_{a,b,\sigma} \rightarrow (LT_{a,c,\sigma} \rightarrow \alpha, \gamma), \beta &= LT_{a,b,\sigma} \rightarrow \alpha, \beta, &&\mbox{con $b\leq c$}
    \end{align*}
    
    Le tre proprietà esposte sopra valgono anche quando il comando $if$ interno fa parte di espressioni o comandi più complessi.\\
    \\
    Banalmente il comando \textbf{noaction} viene interpretato come:
    
    \begin{equation*}
        \mathbb{C}\doublebracket{\mbox{\textbf{noaction}}}\sigma = \sigma
    \end{equation*}
    
    Definiamo adesso l'intero comando $if$ presente, $IF \equiv \mbox{\textbf{if}}\; x<3\; \mbox{\textbf{then}}\; x:=x-1\; \mbox{\textbf{else noaction}}$. La sua interpretazione semantica sarà quindi:
    
    \begin{align*}
        \mathbb{C}\doublebracket{IF}\sigma =\;& \mbox{let} <v_1,\sigma_1> \mbox{be}\; \mathbb{E}\doublebracket{x<3}\sigma\; \mbox{in}\\
        & v_1\rightarrow\mathbb{C}\doublebracket{x:=x-1}\sigma_1,\mathbb{C}\doublebracket{\mbox{\textbf{noaction}}}\sigma_1\\
        =\;& LT_{mx,3,\sigma}\rightarrow \sigma[M_{mx,1,\sigma}/x], \sigma
    \end{align*}
    
    Definiamo ora il funzionale della specifica ricorsiva del comando \textbf{while} in esame, depurato da controlli d'errore, in modo simile a quanto fatto nelle dispense a pag. 187.:
    
    \begin{equation*}
        F \equiv \lambda\Theta.\lambda\sigma.\; \mbox{let} <v_1,\sigma_1> \mbox{be}\; \mathbb{E}\doublebracket{x>0}\sigma\; \mbox{in}\; v_1 \rightarrow \Theta(\mathbb{C}\doublebracket{IF}\sigma_1), \sigma_1
    \end{equation*}
    
    Sostituendo le interpretazioni semantiche già calcolate otteniamo:
    
    \begin{align*}
        F = \lambda\Theta.\lambda\sigma. GT_{mx,0,\sigma} \rightarrow \Theta(LT_{mx,3,\sigma}\rightarrow \sigma[M_{mx,1,\sigma}/x], \sigma), \sigma
    \end{align*}
    
    Per definizione, sappiamo che l'interpretazione del comando \textbf{while} è data da:
    
    \begin{equation*}
        \mathbb{C}\doublebracket{\mbox{\textbf{while}}\; x>0\; \mbox{\textbf{do}}\; \mbox{\textbf{if}}\; x<3\; \mbox{\textbf{then}}\; x:=x-1\; \mbox{\textbf{else noaction}}} = fix(F)
    \end{equation*}
    
    Calcoliamo allora il punto fisso del funzionale $F$ per approssimazioni successive:
    
    \begin{align*}
        F^0\Omega =\;& \Omega\\
        =\;& \lambda x.\bot
    \end{align*}
    
    La prima approssimazione è la meno definita, infatti ci sta dicendo che il comando diverge sempre. Tuttavia noi stiamo cercando un punto fisso, quindi andiamo avanti:
    
    \begin{align*}
        F^1\Omega =\;& F(F^0\Omega)\\
        =\;& F\Omega\\
        =\;& \lambda\sigma. GT_{mx,0,\sigma} \rightarrow \Omega(LT_{mx,3,\sigma}\rightarrow \sigma[M_{mx,1,\sigma}/x], \sigma), \sigma\\
        =\;& \lambda\sigma. GT_{mx,0,\sigma} \rightarrow \bot, \sigma
    \end{align*}
    
    Il comando è già più definito, in quanto adesso sappiamo che non diverge se il valore assegnato ad $x$ non è maggiore di $0$ (comprendendo anche i vari errori che non abbiamo incluso, quali valori booleani assegnati alla $x$ o \textit{unbound}). Vediamo come si comporta la seconda approssimazione:
    
    \begin{align*}
        F^2\Omega =\;& F(F^1\Omega)\\
        =\;& \lambda\sigma. GT_{mx,0,\sigma} \rightarrow (\lambda\sigma_1. GT_{mx,0,\sigma_1} \rightarrow \bot, \sigma_1)(LT_{mx,3,\sigma}\rightarrow \sigma[M_{mx,1,\sigma}/x], \sigma), \sigma\\
        =\;& \lambda\sigma. GT_{mx,0,\sigma} \rightarrow (LT_{mx,3,\sigma}\rightarrow (GT_{mx,1,\sigma} \rightarrow \bot, \sigma[M_{mx,1,\sigma}/x]), (GT_{mx,0,\sigma} \rightarrow \bot, \sigma)), \sigma\\
        =\;& \lambda\sigma. GT_{mx,0,\sigma} \rightarrow (LT_{mx,3,\sigma}\rightarrow (GT_{mx,1,\sigma} \rightarrow \bot, \sigma[M_{mx,1,\sigma}/x]), \bot), \sigma
    \end{align*}
    
    L'interpretazione del comando \textbf{while} inizia a prendere forma: l'approssimazione appena calcolata ci dice che se il valore associato ad $x$ è $2$, allora decrementa il suo valore in memoria, altrimenti o diverge o non fa niente. Calcoliamo la prossima approssimazione:
    
    \begin{align*}
        F^3\Omega =\;& F(F^2\Omega)\\
        =\;& \lambda\sigma. GT_{mx,0,\sigma} \rightarrow (\lambda\sigma_1. GT_{mx,0,\sigma_1} \rightarrow (LT_{mx,3,\sigma_1}\rightarrow (GT_{mx,1,\sigma_1} \rightarrow \bot, \sigma_1[M_{mx,1,\sigma_1}/x]), \bot), \sigma_1)\\
        &(LT_{mx,3,\sigma}\rightarrow \sigma[M_{mx,1,\sigma}/x], \sigma), \sigma\\
        =\;& \lambda\sigma. GT_{mx,0,\sigma} \rightarrow (LT_{mx,3,\sigma}\rightarrow (GT_{mx,0,\sigma[M_{mx,1,\sigma}/x]} \rightarrow (LT_{mx,3,\sigma[M_{mx,1,\sigma}/x]}\rightarrow\\
        &(GT_{mx,1,\sigma[M_{mx,1,\sigma}/x]} \rightarrow \bot, \sigma[M_{mx,1,\sigma}/x][M_{mx,1,\sigma[M_{mx,1,\sigma}/x]}/x]), \bot), \sigma[M_{mx,1,\sigma}/x]),\\
        &(GT_{mx,0,\sigma} \rightarrow (LT_{mx,3,\sigma}\rightarrow (GT_{mx,1,\sigma} \rightarrow \bot, \sigma[M_{mx,1,\sigma}/x]), \bot), \sigma)), \sigma\\
        =\;& \lambda\sigma. GT_{mx,0,\sigma} \rightarrow (LT_{mx,3,\sigma}\rightarrow (GT_{mx,1,\sigma} \rightarrow \sigma[M_{mx,2,\sigma}/x], \sigma[M_{mx,1,\sigma}/x]), \bot), \sigma
    \end{align*}
    
    Questa terza approssimazione è molto interessante! Quello che ci dice è che per $0$ il comando non fa niente, per valori maggiori o uguali a $3$ diverge senza modificare alcun valore, altrimenti decrementa $x$ di $2$ o di $1$ a seconda che il valore di $x$ sia $2$ o $1$, rispettivamente. Intuitivamente, questa è proprio la descrizione che ci viene in mente quando pensiamo all'interpretazione di questo comando \textbf{while}.
    
    \begin{align*}
        F^4\Omega =\;& F(F^3\Omega)\\
        =\;& \lambda\sigma. GT_{mx,0,\sigma} \rightarrow (\lambda\sigma_1. GT_{mx,0,\sigma_1} \rightarrow (LT_{mx,3,\sigma_1}\rightarrow (GT_{mx,1,\sigma_1} \rightarrow \sigma_1[M_{mx,2,\sigma_1}/x],\\
        &\sigma_1[M_{mx,1,\sigma_1}/x]), \bot), \sigma_1)(LT_{mx,3,\sigma}\rightarrow \sigma[M_{mx,1,\sigma}/x], \sigma), \sigma\\
        =\;& \lambda\sigma. GT_{mx,0,\sigma} \rightarrow (LT_{mx,3,\sigma}\rightarrow (GT_{mx,1,\sigma} \rightarrow (LT_{mx,4,\sigma}\rightarrow\\
        &(GT_{mx,2,\sigma} \rightarrow \sigma[M_{mx,3,\sigma}/x],\sigma[M_{mx,2,\sigma}/x]), \bot), \sigma[M_{mx,1,\sigma}/x]),(GT_{mx,0,\sigma} \rightarrow\\
        &(LT_{mx,3,\sigma}\rightarrow (GT_{mx,1,\sigma} \rightarrow \sigma[M_{mx,2,\sigma}/x],\sigma[M_{mx,1,\sigma}/x]), \bot), \sigma)), \sigma\\
        =\;& \lambda\sigma. GT_{mx,0,\sigma} \rightarrow (LT_{mx,3,\sigma}\rightarrow (GT_{mx,1,\sigma} \rightarrow \sigma[M_{mx,2,\sigma}/x], \sigma[M_{mx,1,\sigma}/x]), \bot), \sigma\\
        =\;& F^3\Omega
    \end{align*}
    
    Abbiamo trovato un punto fisso! La terza approssimazione calcolata prima risulta essere un punto fisso (il minimo in particolare) in quanto $F(F^3\Omega)=F^3\Omega$. Questo ci dice che la descrizione informale data prima del comando è \textit{esattamente} l'interpretazione del comando e non una semplice approssimazione.\\
    \\
    Quindi, ricapitolando, possiamo dedurre che l'insieme degli stati che fanno divergere il comando proposto sono tutti queli stati che fanno finire l'interpretazione semantica del \textbf{while} nel ramo contenente $\bot$, ovvero tutti quegli stati con input e output qualsiasi e con valori in memoria associati alla variabile $x$ maggiori o uguali a $3$. Come già detto all'inizio, è facile vedere come i valori per $x$ che causano errore, quali valori booleani o \textit{unbound}, garantiscano la convergenza, in quanto generano un errore che viene subito propagato, bloccando l'esecuzione del programma.
