\section*{Esercizio 4.10}

    \enunciato{Si dimostrino le seguenti uguaglianze usando il sistema di inferenza della semantica assiomatica:
        \begin{align*}
            c)& \; E=F \Longrightarrow E^* = F^*\\
            d)& \; E^* +1=E^*\\
            e)& \; E^* + E = E^*
        \end{align*}
    }
    
    Sfruttando gli assiomi e le due regole in Tabella 4.4 delle dispense, si dimostrano le uguaglianze proposte.\\
    
    \begin{description}
        \item[$c) \; E=F \Longrightarrow E^* = F^*$] \hfill \\
            Come già mostrato nelle dispense, si può dimostrare che l'uguaglianza $=$ è riflessiva e simmetrica:
            \begin{equation}
                \inferrule* [Right={$(regola1)$}]
                    {E+E=E \\ E+E=E}
                    {E=E \\ E=E+E}
                \label{eq:riflessività}
            \end{equation}
            \begin{equation}
                \inferrule* [Right={$(regola1)$}]
                    {E=F \\ E=F}
                    {F=F \\ F=E}
                \label{eq:simmetria}
            \end{equation}
            Da qui in avanti useremo i risultati della (\ref{eq:riflessività}) e della (\ref{eq:simmetria}) come fossero regole, indicandole con $(rifl)$ e $(simm)$, rispettivamente.\\
            \\
            Per dimostrare quindi che $E=F \Longrightarrow E^* = F^*$ basta applicare la regola della sostituzione assumendo riflessività e simmetria della relazione $=$.
            \begin{equation}
                \inferrule* [Right={$(simm)$}]
                    {\inferrule* [Right={$(regola1)$}]
                        {E=F \\ \inferrule* [Right={$(rifl)$}]
                            { }
                            {E^* = E^*}
                        }
                        {F^*=E^*}
                    }
                    {E^* = F^*}
            \end{equation}
        \item[$d) \; E^* +1=E^*$] \hfill \\
            Dimostriamo innanzitutto la sostitutività rispetto alla somma (che indicheremo con $(sost+)$) come suggerito dalla Proposizione 4.13 delle dispense:
            \begin{equation}
                \inferrule* [Right={$(simm)$}]
                    {\inferrule* [Right={$(regola1)$}]
                        {G=H \\ \inferrule* [Right={$(regola1)$}]
                            {E=F \\ \inferrule* [Right={$(rifl)$}]
                                { }
                                {E+G=E+G}
                            }
                            {F+G=E+G}
                        }
                        {F+H=E+G}
                    }
                    {E+G=F+H}
            \end{equation}
            La dimostrazione che $E^*+1=E^*$ si ottiene quindi come segue (abbiamo diviso la dimostrazione in quattro parti per questioni di leggibilità):
            \begin{equation}
                \scriptsize
                \inferrule* [Right={$(simm)$}]
                    {\inferrule* [Right={$(sost+)$}]
                        {\inferrule* [right={$(unfolding)$}]
                            { }
                            {E^*=1+E^*E}\\
                        \inferrule* [Right=$(rifl)$]
                            { }
                            {1=1}
                        }
                        {E^*+1=1+E^*E+1}
                    }
                    {1+E^*E+1=E^*+1}
                \label{eq:1+E^*E+1=E^*+1}
            \end{equation}
            \begin{equation}
                \scriptsize
                \inferrule* [Right={$(regola1)$}]
                    {\inferrule* [right={$(comm+)$}]
                        { }
                        {1+E^*E+1=1+1+E^*E}\\
                    \inferrule* [Right=(\ref{eq:1+E^*E+1=E^*+1})]
                        { }
                        {1+E^*E+1=E^*+1}
                    }
                    {1+1+E^*E=E^*+1}
                \label{eq:1+1+E^*E=E^*+1}
            \end{equation}
            \begin{equation}
                \scriptsize
                \inferrule* [Right={$(regola1)$}]
                    {\inferrule* [right={$(idem+)$}]
                        { }
                        {1+1=1}\\
                    \inferrule* [Right=(\ref{eq:1+1+E^*E=E^*+1})]
                        { }
                        {1+1+E^*E=E^*+1}
                    }
                    {1+E^*E=E^*+1}
                \label{eq:1+E^*E=E^*+1}
            \end{equation}
            \begin{equation}
                \scriptsize
                \inferrule* [Right={$(simm)$}]
                    {\inferrule* [Right={$(regola1)$}]
                        {\inferrule* [right={$(simm)$}]
                            {\inferrule* [right={$(unfolding)$}]
                                { }
                                {E^*=1+E^*E}
                            }
                            {1+E^*E=E^*}\\
                        \inferrule* [Right=(\ref{eq:1+E^*E=E^*+1})]
                            { }
                            {1+E^*E=E^*+1}
                        }
                        {E^*=E^*+1}
                    }
                    {E^*+1=E^*}
                \label{eq:E^*+1=E^*}
            \end{equation}
            In parole povere, si è fatto un primo passo di unfolding di $E^*$, ci si è sommato $1$ e si è fatto vedere che i due $1$ si riducono ad un solo $1$. Quindi abbiamo rimesso le cose a posto per avere la formulazione finale richiesta.
        \item[$e) \; E^* + E = E^*$] \hfill \\
            L'idea di quest'ultima dimostrazione è sulla falsariga della dimostrazione precedente, ovvero si dimostra facendo un passo di unfolding e poi rimettendo tutto a posto. Ad un certo punto sarà molto utile anche sfruttare il risultato della dimostrazione precedente.
            \begin{equation}
                \scriptsize
                \inferrule* [Right={$(simm)$}]
                    {\inferrule* [Right={$(sost+)$}]
                        {\inferrule* [right={$(unfolding)$}]
                            { }
                            {E^*=1+E^*E}\\
                        \inferrule* [Right={$(simm)$}]
                            {\inferrule* [Right={$(neutro;)$}]
                                { }
                                {1E=E}
                            }
                            {E=1E}
                        }
                        {E^*+E=1+E^*E+1E}
                    }
                    {1+E^*E+1E=E^*+E}
                \label{eq:1+E^*E+1E=E^*+E}
            \end{equation}
            \begin{equation}
                \scriptsize
                \inferrule* [Right={$(regola1)$}]
                    {\inferrule* [right={$(distribD)$}]
                        { }
                        {E^*E+1E=(E^*+1)E}\\
                    \inferrule* [Right=(\ref{eq:1+E^*E+1E=E^*+E})]
                        { }
                        {1+E^*E+1E=E^*+E}
                    }
                    {1+(E^*+1)E=E^*+E}
                \label{eq:1+(E^*+1)E=E^*+E}
            \end{equation}
            \begin{equation}
                \scriptsize
                \inferrule* [Right={$(regola1)$}]
                    {\inferrule* [right=(\ref{eq:E^*+1=E^*})]
                        { }
                        {E^*+1=E^*}\\
                    \inferrule* [Right=(\ref{eq:1+(E^*+1)E=E^*+E})]
                        { }
                        {1+(E^*+1)E=E^*+E}
                    }
                    {1+E^*E=E^*+E}
                \label{eq:1+E^*E=E^*+E}
            \end{equation}
            \begin{equation}
                \scriptsize
                \inferrule* [Right={$(simm)$}]
                    {\inferrule* [Right={$(regola1)$}]
                        {\inferrule* [right={$(simm)$}]
                            {\inferrule* [right={$(unfolding)$}]
                                { }
                                {E^*=1+E^*E}
                            }
                            {1+E^*E=E^*}\\
                        \inferrule* [Right=(\ref{eq:1+E^*E=E^*+E})]
                            { }
                            {1+E^*E=E^*+E}
                        }
                        {E^*=E^*+E}
                    }
                    {E^*+E=E^*}
                \label{eq:E^*+E=E^*}
            \end{equation}
    \end{description}
