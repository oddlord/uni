\section*{Esercizio 3.15}

    \enunciato{0.8}{
        Dimostrare che, per ogni espressione aritmetica $E$ descritta nella Sezione 3.3,\\
        \begin{center}
            $E \rightarrow E_1, E \rightarrow E_2 \;\mbox{ implica }\; E_1 \twoheadrightarrow n, E_2 \twoheadrightarrow n.$
        \end{center}
    }
    
    Quello che si richiede in questo esercizio è di dimostrare che quando una stessa espresione aritmetica $E$ (generata tramite la grammatica in Tabella 3.3) viene \textit{computata} in due modi distinti, producendo le nuove espressioni $E_1$ ed $E_2$, allora possiamo dire che queste due nuove espressioni vengono \textit{valutate} entrambe lo stesso numero $n$.\\
    \\
    Ragioniamo per induzione sul numero di operatori in $E$:
    \begin{itemize}
        \item \textbf{Caso base:}\\
            Supponiamo che in $E$ ci sia un solo operatore (se non ci fossero operatori allora $E$ sarebbe composto da un solo numero e non potrebbe essere computato in un'altra espressione, ma soltanto valutato), allora l'unica regola applicabile è \textit{(op)}. Dal fatto che l'unica regola applicabile è \textit{(op)}, segue che $E_1=E_2=n$, per un $n$ opportuno. Allora per la prima regola in Tabella 3.5 si ottiene banalmente:
            $$E_1 = n \twoheadrightarrow n$$
            $$E_2 = n \twoheadrightarrow n$$
        \item \textbf{Passo induttivo:}\\
            Supponiamo che in $E$ ci siano $k+1$ operatori, con $k \geq 1$, e che la proposizione valga per tutte le espressioni con $k$ operatori. Allora $E$ sarà della forma $E=E_a\; op\; E_b$, per $E_a$, $E_b$ e $op$ opportuni, e potrà essere computata tramite \textit{(redl)} oppure \textit{(redr)}.\\
            \\
            Se $E_1$ ed $E_2$ sono state computate tramite la stessa regola, allora per quanto detto prima esse risultano uguali e banalmente vengono valutate nello stesso numero.\\
            Supponiamo allora di aver ottenuto $E_1$ tramite \textit{(redl)} e $E_2$ tramite \textit{(redr)}. Questo significa che se $E_a \rightarrow E_a'$ ed $E_b \rightarrow E_b'$, allora $E_1$ ed $E_2$ hanno la forma:
            $$E_1 = E_a'\; op\; E_b$$
            $$E_2 = E_a\; op\; E_b'$$
            Possiamo osservare come ogni computazione tramite le tre regole fornite implichi sempre il decremento unitario del numero di operatori rispetto all'espressione originale. Ovvero ogni regola elimina sempre uno ed un solo operatore. Questo implica che la proposizione risulta vera, per ipotesi induttiva, per $E_1$, $E_2$, $E_a$, $E_b$, $E_a'$ e $E_b'$. Questo ci assicura che, indipendentemente da quali regole di computazione verranno utilizzate, queste espressioni verranno valutate sempre nello stesso numero.\\
            Possiamo quindi dire che $E_a' \twoheadrightarrow x$ e $E_b' \twoheadrightarrow y$, per certi $x$ e $y$.\\
            \\
            Sfruttando adesso l'equivalenza tra semantica di computazione e semantica di valutazione\footnote{Questo risultato verrà dato per buono ai fini di questo esercizio in quanto la sua dimostrazione è già obiettivo di un altro esercizio, l'Esercizio 3.14.} notiamo che, avendo $E_a' \twoheadrightarrow x$, allora $E_a \xrightarrow{*} x$. Ricordiamo anche che per definizione $E_a \rightarrow E_a'$, ovvero $E_a \xrightarrow{*} x$ e quindi, applicando nuovamente l'equivalenza tra le due semantiche, otteniamo che $E_a \twoheadrightarrow x$. Ovvero sia $E_a$ che $E_a'$ vengono valutate entrambe $x$.\\
            Lo stesso discorso si applica ad $E_b$ e $E_b'$, ricavando che entrambi sono valutati in $y$.\\
            \\
            Sia quindi $x\; op\; y = n$, allora
            \begin{equation*}
                \inferrule* [Right={$(x\; op\; y = n)$}]
                    {E_a' \twoheadrightarrow x \\ E_b \twoheadrightarrow y}
                    {E_a'\; op\; E_b \twoheadrightarrow n}
            \end{equation*}
            \begin{equation*}
                \inferrule* [Right={$(x\; op\; y = n)$}]
                    {E_a \twoheadrightarrow x \\ E_b' \twoheadrightarrow y}
                    {E_a\; op\; E_b' \twoheadrightarrow n}
            \end{equation*}
            da cui la tesi ricordando la forma di $E_1$ ed $E_2$.\\
            \\
            Se, alternativamente, $E_1$ fosse stato ottenuto tramite \textit{(redr)} ed $E_2$ tramite \textit{(redl)} i passaggi sarebbero stati del tutto analoghi.
    \end{itemize}
