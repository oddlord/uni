\documentclass[9pt]{beamer}

\usepackage{amsmath}
\usepackage{amsfonts}
\usepackage{amssymb}
\usepackage{graphicx}

\usepackage[italian]{babel}
\usepackage[latin1]{inputenc}
\usepackage[T1]{fontenc}
\usepackage{times}

\usetheme{Antibes} % Antibes, Berkeley, Hannover, Luebeck
\usecolortheme{dolphin} % beaver, dolphin
\setbeamercovered{dynamic}
\newcommand{\nologo}{\setbeamertemplate{logo}{}} % command to set the logo to nothing

\setbeamertemplate{navigation symbols}{}
\setbeamertemplate{footline}[frame number]

\title[Riordino intelligente tramite MDP]{Riordino intelligente dei farmaci tramite modelli MDP}

\author{M.Biagi \and \textbf{T.Papini} \and}

\institute{
    STLab, Dipartimento d'Ingegneria dell'Informazione, Universit� degli Studi di Firenze, Italia,\\
    {\{marco.biagi,tommaso.papini\}@unifi.it}}

\date{31 Marzo 2017}

\setbeamertemplate{footline}[frame number]
\setbeamertemplate{navigation symbols}{}



\begin{document}

    \begin{frame}
        \titlepage
        \begin{itemize}
            \item Tecniche basate su MDP
            \begin{itemize}
                \item modellazione del reparto
                \item attualizzazione del modello
                \item sperimentazione
            \end{itemize}
        \end{itemize}
    \end{frame}

    \begin{frame}{Panoramica}
        %\tiny
        \tableofcontents
    \end{frame}

    \section{Il modello generale}
    
        \subsection{Struttura del modello}
            \begin{frame}{Struttura del modello}
            \end{frame}
        
        \subsection{Specifiche e limitazioni}
            \begin{frame}{Specifiche}
            \end{frame}
            
            \begin{frame}{Limitazioni}
            \end{frame}
    
    \section{Utilizzo del modello}
    
        \subsection{Strumenti utilizzati}
            \begin{frame}{Strumenti utilizzati}
            \end{frame}
    
        \subsection{Attualizzazione del modello}
            \begin{frame}{Attualizzazione del modello}
            \end{frame}
            
        \subsection{Estrazione dell'avversario}
            \begin{frame}{Estrazione dell'avversario}
            \end{frame}
            
    \section{Sperimentazione}
    
        \begin{frame}{Sperimentazione}
        \end{frame}

\end{document}
