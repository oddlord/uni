% !TeX program = pdflatex
% !TeX encoding = UTF-8
% !TeX spellcheck = it_IT

\documentclass[9pt]{beamer}

% !TeX root = ../aal-modelos-transparencias.tex
% !TeX encoding = UTF-8
% !TeX spellcheck = es_ES

%********************************************************************
% Packages
%********************************************************************

\usepackage[spanish]{babel}
\usepackage[utf8]{inputenc}
\usepackage[T1]{fontenc}

\usepackage{amsmath}
\usepackage{amsfonts}
\usepackage{amssymb}
\usepackage{calligra}
\usepackage{graphicx}
\usepackage{tikz}
\usetikzlibrary{arrows,automata,shapes,calc,backgrounds,positioning}
\usepackage{times}

\graphicspath{{img/}}

%********************************************************************
% Beamer styles
%********************************************************************

\usetheme{Montpellier}
\usecolortheme{default}
\setbeamercovered{dynamic}
\newcommand{\nologo}{\setbeamertemplate{logo}{}} % command to set the logo to nothing

\setbeamertemplate{navigation symbols}{}
\setbeamertemplate{footline}[frame number]

\setbeamertemplate{footline}[frame number]
\setbeamertemplate{navigation symbols}{}

\usebackgroundtemplate{
  \begin{tikzpicture}
    \node[opacity=0.05] {\includegraphics[]{logoUnifi.png}};
  \end{tikzpicture}
}

\AtBeginSection[]{
  \begin{frame}
  \vfill
  \centering
  \begin{beamercolorbox}[sep=8pt,center,shadow=true,rounded=true]{title}
    \usebeamerfont{title}\insertsectionhead\par%
  \end{beamercolorbox}
  \vfill
  \end{frame}
}


\title[Riordino intelligente tramite MDP]{Riordino intelligente dei farmaci tramite modelli MDP}
\author{M.Biagi \and \textbf{T.Papini} \and}
\institute{
    STLab, Dipartimento d'Ingegneria dell'Informazione, Università degli Studi di Firenze, Italia,\\
    {\{marco.biagi,tommaso.papini\}@unifi.it}
}
\date{31 Marzo 2017}

\begin{document}

    \begin{frame}
        \titlepage
        \begin{itemize}
            \item Tecniche basate su MDP
            \begin{itemize}
                \item modellazione del reparto
                \item attualizzazione del modello
                \item sperimentazione
            \end{itemize}
        \end{itemize}
    \end{frame}

    \begin{frame}{Panoramica}
        %\tiny
        \tableofcontents
    \end{frame}

    \section{Il modello generale}
    
        \subsection{Struttura del modello}
            \begin{frame}{Struttura del modello}
              \begin{itemize}
                \item Modellazione di più giorni consecutivi
                \begin{itemize}
                  \item giorno corrente (fine giornata)
                  \item 3 giorni successivi
                \end{itemize}
                \item probabilità e non-determinismo
                \begin{itemize}
                  \item evoluzione probabilistica di ogni giornata
                  \item scelte non-deterministiche sull'ordine
                \end{itemize}
              \end{itemize}
              \begin{center}
                \begin{tikzpicture}[node distance = 2cm, auto]
                  % Place nodes
                  \node [block] at (0,0) (oggi) [] {Giorno\\corrente};
                  \node [block] at (2.5,0) (g1) [] {Giorno 1};
                  \node [block] at (5,0) (g2) [] {Giorno 2};
                  \node [block] at (7.5,0) (g3) [] {Giorno 3};
                  % Draw edges
                  \path [line] (oggi) -- (g1);
                  \path [line] (g1) -- (g2);
                  \path [line] (g2) -- (g3);
                \end{tikzpicture}
              \end{center}
            \end{frame}
        
        \subsection{Specifiche e limitazioni}
            \begin{frame}{Specifiche}
            \end{frame}
            
            \begin{frame}{Limitazioni}
            \end{frame}
    
    \section{Utilizzo del modello}
    
        \subsection{Strumenti utilizzati}
            \begin{frame}{Strumenti utilizzati}
            \end{frame}
    
        \subsection{Attualizzazione del modello}
            \begin{frame}{Attualizzazione del modello}
            \end{frame}
            
        \subsection{Estrazione dell'avversario}
            \begin{frame}{Estrazione dell'avversario}
            \end{frame}
            
    \section{Sperimentazione}
    
        \begin{frame}{Sperimentazione}
        \end{frame}

\end{document}
