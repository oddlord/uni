% !TeX spellcheck = it_IT
\documentclass{article}
\usepackage[utf8]{inputenc}
\usepackage[T1]{fontenc}
\usepackage[italian]{babel}
\usepackage{graphicx}
\usepackage{hyperref}
\usepackage{xcolor}
\hypersetup{
    colorlinks,
    linkcolor={red!50!black},
    citecolor={blue!50!black},
    urlcolor={blue!80!black}
}
\graphicspath{{img/}}

\begin{document}
    \pagenumbering{gobble}
    
    \begin{center}
        \includegraphics[scale=0.2]{logo_unifi.jpg}\\[4cm]
        \textit{Relazione di fine Assegno di Ricerca}\\[0.3cm]
        {\Huge Metodi quantitativi per l'analisi di percorsi clinici -RACE}\\[0.2cm]
        Progetto finanziato nel quadro del POR FESR Toscana 2014-2020
    \end{center}
    
    \vfill
    
    \textbf{Ricercatore}: Tommaso \textsc{Papini}
    
    \clearpage
    
    L'assegno di ricerca ``Metodi quantitativi per l'analisi di percorsi clinici -RACE'', oggetto di questa relazione, è stato assegnato al sottoscritto, Tommaso Papini, a partire dal 01/06/2016 per una durata prevista di 12 mesi. È stata tuttavia fatta richiesta ufficiale di rinuncia (indicando come ultimo giorno di lavoro il 31/10/2016) in quanto il sottoscritto è risultato vincitore di un posto di dottorato con borsa di studio per il programma di dottorato in Smart Computing presso l'Università di Firenze. Il lavoro che verrà svolto durante suddetto dottorato sarà tuttavia inerente e di carattere continuativo rispetto al lavoro svolto durante questi primi 5 mesi di assegno di ricerca.
    
    Il primo progetto seguito durante l'assegno di ricerca è stata la stesura di un articolo da presentare alla conferenza annuale MASCOTS (Modelling, Analysis and Simulation of Computer and Telecommunication Systems)\footnote{\url{http://san.ee.ic.ac.uk/mascots2016/}}, tenutasi dal 19 al 21 settembre 2016 presso l'Imperial College di Londra, Regno Unito. L'articolo \cite{mascots16}, dal titolo \textit{Performance Evaluation of Fischer's Protocol through Steady-State Analysis of Markov Regenerative Processes}, aveva il duplice obiettivo di illustrare una nuova tecnica per l'analisi steady-state di processi di Markov rigenerativi (MRP, Markov Regenerative Processes), da un lato, ed applicare la suddetta tecnica ad uno scenario di esempio di analisi delle performance di un sistema concorrente, dall'altro. Il caso di studio preso in considerazione è quello del protocollo di mutua esclusione di Fischer
    
    
    
    
    \clearpage
    
	\bibliographystyle{IEEEtran}
	\bibliography{bibliography}
\end{document}