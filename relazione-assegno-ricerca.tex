% !TeX spellcheck = it_IT
\documentclass{article}
\usepackage[utf8]{inputenc}
\usepackage[T1]{fontenc}
\usepackage[italian]{babel}
\usepackage{graphicx}
\usepackage{hyperref}
\usepackage{xcolor}
\hypersetup{
    colorlinks,
    linkcolor={red!50!black},
    citecolor={blue!50!black},
    urlcolor={blue!80!black}
}
\graphicspath{{img/}}

\begin{document}
    \pagenumbering{gobble}
    
    \begin{center}
        \includegraphics[scale=0.2]{logo_unifi.jpg}\\[4cm]
        \textit{Relazione di fine Assegno di Ricerca}\\[0.3cm]
        {\Huge Metodi quantitativi per l'analisi di percorsi clinici -RACE}\\[0.2cm]
        Progetto finanziato nel quadro del POR FESR Toscana 2014-2020
    \end{center}
    
    \vfill
    
    \textbf{Ricercatore}: Tommaso \textsc{Papini}
    
    \clearpage
    
    L'assegno di ricerca ``Metodi quantitativi per l'analisi di percorsi clinici -RACE'', oggetto di questa relazione, è stato assegnato al sottoscritto, Tommaso Papini, a partire dal 01/06/2016 per una durata prevista di 12 mesi. È stata tuttavia fatta richiesta ufficiale di rinuncia (indicando come ultimo giorno di lavoro il 31/10/2016) in quanto il sottoscritto è risultato vincitore di un posto di dottorato con borsa di studio per il programma di dottorato in Smart Computing presso l'Università di Firenze. Il lavoro che verrà svolto durante suddetto dottorato sarà tuttavia inerente e di carattere continuativo rispetto al lavoro svolto durante questi primi 5 mesi di assegno di ricerca.
    
    Nel corso di questi 5 mesi di assegno di ricerca sono stati studiati e sperimentati metodi di valutazione quantitativa applicati all'analisi quantitativa di percorsi clinici e finalizzati all'identificazione precoce di condizioni di non aderenza al percorso di cura. La ricerca è stata caratterizzata dalla combinazione di metodi formali e quantitativi con aspetti di progettazione e sviluppo di applicazioni software, finalizzata alla sperimentazione dei metodi in congiunzione con applicazioni software in uso.
    
    Dal punto di vista funzionale, la ricerca ha studiato lo sviluppo di un metodo per identificare se il percorso clinico effettivo di un paziente è aderente o meno con i protocolli e linee guida specificati. In particolare sono stati studiati metodi per situazioni in cui alcune parti della fase di diagnostica ed alcuni eventi terapeutici non sono totalmente osservabili.
    
    Dal punto di vista strutturale, la ricerca ha studiato l'applicazione di metodi formali e quantitativi basati su modelli misti (stocastici e nondeterministici) utilizzando formalismi di modellazione come le reti di Petri stocastiche o temporizzate.
    
    A supporto della ricerca scientifica, necessaria per lo sviluppo di un metodo per l'analisi di percorsi clinici, è stata studiata più nel dettaglio la teoria sottostante ai processi rigenerativi di Markov (MRP, Markov Regenerative Processes) ed al calcolo delle probabilità a regime per tali processi stocastici. Questa prima fase si è conclusa con la stesura di un articolo da presentare alla conferenza annuale MASCOTS (Modelling, Analysis and Simulation of Computer and Telecommunication Systems)\footnote{\url{http://san.ee.ic.ac.uk/mascots2016/}}, tenutasi dal 19 al 21 settembre 2016 presso l'Imperial College di Londra, Regno Unito. L'articolo \cite{mascots16}, dal titolo \textit{Performance Evaluation of Fischer's Protocol through Steady-State Analysis of Markov Regenerative Processes}, aveva il duplice obiettivo di illustrare una nuova tecnica per l'analisi steady-state di processi di Markov rigenerativi, da un lato, ed applicare la suddetta tecnica ad uno scenario di esempio di analisi delle performance di un sistema concorrente, dall'altro. Il caso di studio preso in considerazione è quello del protocollo di mutua esclusione di Fischer \cite{fischer85}, all'interno del quale più processi tentano ripetutamente di accedere ad una risorsa condivisa utilizzando un metodo di accesso alla sezione critica distribuito e basato sul tempo. Dopo un'introduzione sulla teoria di base sul protocollo di Fischer e i MRP, l'articolo studia come è influenzata la latenza di accesso alla sezione critica per un singolo processo in funzione di vari parametri caratteristici del modello. Nella sezione sperimentale dell'articolo si sono quindi condotti diversi esperimenti nei quali si variano i parametri del modello e, per ogni configurazione di parametri, se ne calcola la latenza, andando quindi a rappresentare i risultati sotto forma grafica. La tecnica per analizzare lo steady-state dei MRP è stata precedentemente implementata, ed inclusa all'interno della libreria Sirio (sviluppata da STLab, Dinfo, UniFi), dal sottoscritto e dal collega Stefano Martina come progetto per l'esame del corso magistrale di Metodi di Verifica e Testing. L'articolo è stato accettato come short paper da MASCOTS e presentato alla conferenza in data 21 settembre 2016.
    
    Dal punto di vista strutturale, invece, si è visto necessario lavorare all'implementazione in Oris \cite{bucci2010oris}, un programma per l'analisi di reti di Petri temporizzate e stocastiche sviluppato da STLab, di una nuova tecnica di analisi transiente ed in particolare l'analisi transiente di MRP sotto enabling restriction \cite{german1995transient}. Questo nuovo tipo di analisi ha ampliato il potere di analisi del software Oris ed ha, allo stesso tempo, aperto nuove strade per l'analisi quantitativa di MRP, specialmente in ambito clinico. Con questo nuovo tipo di analisi transiente sarà infatti possibile effettuare tutta una serie di analisi delle probabilità transienti di un sistema ma in modo molto efficiente e veloce.
    
    Infine, sono stati fatti dei lavori di aggiornamento ed ampliamento del sito internet di Oris. Il sito, raggiungibile all'indirizzo \url{http://www.oris-tool.org/}, era precedentemente costituito da un'unica pagina, contenente alcune informazioni di base su come scaricarlo, sulle pubblicazioni relative e sui partecipanti al progetto Oris nel corso degli anni. Con questo aggiornamento si sono innanzitutto create delle pagine dedicate ad ogni sezione, raggiungibili tramite una barra di navigazione posta in alto alla pagina, ma soprattutto è stata aggiunta una sezione di tutorial dove viene mostrato l'utilizzo del programma, passo per passo, tramite alcuni esempi di base, in modo da guidare un nuovo utente durante i primi utilizzi di Oris.
    
    \clearpage
    
	\bibliographystyle{IEEEtran}
	\bibliography{bibliography}
\end{document}