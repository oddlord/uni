% !TeX spellcheck = it_IT
\documentclass{article}
\usepackage[utf8]{inputenc}
\usepackage[T1]{fontenc}
\usepackage[italian]{babel}
\usepackage{graphicx}
\usepackage{hyperref}
\usepackage{xcolor}
\hypersetup{
    colorlinks,
    linkcolor={red!50!black},
    citecolor={blue!50!black},
    urlcolor={blue!80!black}
}
\graphicspath{{img/}}

\begin{document}
    \pagenumbering{gobble}
    
    \begin{center}
        \includegraphics[scale=0.2]{logo_unifi.jpg}\\[4cm]
        \textit{Relazione di fine Assegno di Ricerca}\\[0.3cm]
        {\Huge Metodi quantitativi per l'analisi di percorsi clinici -RACE}\\[0.2cm]
        Progetto finanziato nel quadro del POR FESR Toscana 2014-2020
    \end{center}
    
    \vfill
    
    \textbf{Ricercatore}: Tommaso \textsc{Papini}
    
    \clearpage
    
    L'assegno di ricerca ``Metodi quantitativi per l'analisi di percorsi clinici -RACE'', oggetto di questa relazione, è stato assegnato al sottoscritto, Tommaso Papini, a partire dal 01/06/2016 per una durata prevista di 12 mesi. È stata tuttavia fatta richiesta ufficiale di rinuncia (indicando come ultimo giorno di lavoro il 31/10/2016) in quanto il sottoscritto è risultato vincitore di un posto di dottorato con borsa di studio per il programma di dottorato in Smart Computing presso l'Università di Firenze. Il lavoro che verrà svolto durante suddetto dottorato sarà tuttavia inerente e di carattere continuativo rispetto al lavoro svolto durante questi primi 5 mesi di assegno di ricerca.
    
    Il primo progetto seguito durante l'assegno di ricerca è stata la stesura di un articolo da presentare alla conferenza annuale MASCOTS (Modelling, Analysis and Simulation of Computer and Telecommunication Systems)\footnote{\url{http://san.ee.ic.ac.uk/mascots2016/}}, tenutasi dal 19 al 21 settembre 2016 presso l'Imperial College di Londra, Regno Unito. L'articolo \cite{mascots16}, dal titolo \textit{Performance Evaluation of Fischer's Protocol through Steady-State Analysis of Markov Regenerative Processes}, aveva il duplice obiettivo di illustrare una nuova tecnica per l'analisi steady-state di processi di Markov rigenerativi (MRP, Markov Regenerative Processes), da un lato, ed applicare la suddetta tecnica ad uno scenario di esempio di analisi delle performance di un sistema concorrente, dall'altro. Il caso di studio preso in considerazione è quello del protocollo di mutua esclusione di Fischer \cite{fischer85}, all'interno del quale più processi tentano ripetutamente di accedere ad una risorsa condivisa utilizzando un metodo di accesso alla sezione critica distribuito e basato sul tempo. Dopo un'introduzione sulla teoria di base sul protocollo di Fischer e i MRP, l'articolo studia come è influenzata la latenza di accesso alla sezione critica per un singolo processo in funzione di vari parametri caratteristici del modello. Nella sezione sperimentale dell'articolo si sono quindi condotti diversi esperimenti nei quali si variano i parametri del modello e, per ogni configurazione di parametri, se ne calcola la latenza, andando quindi a rappresentare i risultati sotto forma grafica. La tecnica per analizzare lo steady-state dei MRP è stata precedentemente implementata, ed inclusa all'interno della libreria Sirio (sviluppata da STLab, Dinfo, UniFi), dal sottoscritto e dal collega Stefano Martina come progetto per l'esame del corso magistrale di Metodi di Verifica. L'articolo è stato accettato come short paper da MASCOTS e presentato alla conferenza in data 21 settembre 2016.
    
    Il secondo progetto seguito durante l'assegno di ricerca è stato invece incentrato sull'aggiornamento del sito web del programma Oris. Oris è un programma per l'analisi di reti di Petri temporizzate e stocastiche, sviluppato da STLab. Il sito, raggiungibile all'indirizzo \url{http://www.oris-tool.org/}, era precedentemente costituito da un'unica pagina, contenente alcune informazioni di base su come scaricarlo, sulle pubblicazioni relative e sui partecipanti al progetto Oris nel corso degli anni. Con questo aggiornamento si sono innanzitutto create delle pagine dedicate ad ogni sezione, raggiungibili tramite una barra di navigazione posta in alto alla pagina, ma soprattutto è stata aggiunta una sezione di tutorial dove viene mostrato l'utilizzo del programma, passo per passo, tramite alcuni esempi di base, in modo da guidare un nuovo utente durante i primi utilizzi di Oris.
    
    
    
    
    \clearpage
    
	\bibliographystyle{IEEEtran}
	\bibliography{bibliography}
\end{document}