% !TeX spellcheck = en_GB
\documentclass{article}
\usepackage{graphicx}
\setlength{\parindent}{0pt}% Remove paragraph indent
\begin{document}
    \hspace*{0.5\linewidth}
    \pagenumbering{gobble}
    
    To the honourable Members of the Selection Committee at the IMT Institute, \par \bigskip
    
    I'm writing this letter to you as I would like to apply for the Computer Science and System
    Engineering (CSSE) PhD programme offered at IMT. \par \bigskip
    
    I've been studying Computer Science since I enrolled at the University of Florence back in 2009. Since then, I studied and followed this passion of mine without a pause, as I focused mainly on my academic career, but also on many other projects of various nature but always related to the vast world of Computer Science. I received my Bachelor's Degree in Computer Science (Laurea Triennale in Informatica) back in 2012 and my Master's Degree in Computer Science (Laurea Magistrale in Informatica) recently, in April 2016, both with maximum score and honours (Magna Cum Laude). \par \bigskip
    
    Computer Science is not my only passion. I also love to travel to new places, see and learn new things and, in general, work in international and innovative work places. In these last years I managed to reconcile these two passions of mine through several ways. I studied one year abroad, in Madrid, through the Erasmus Programme, and later I worked one year at CERN, Geneva. Through these experiences I managed to acquired various skills, like learning to communicate on a professional level in a foreign language or learning to live both professional and every day life in a completely new environment, not to mention all the new technologies and scientific notions I have learned and added to my ``cultural baggage''. \par \bigskip
    
    Both the bachelor's and master's thesis were related to the field of Web Development and were supervised by Professor Pierluigi Crescenzi, at University of Florence. In particular for the bachelor's thesis, titled ``Algorithm Visualization in HTML5'', I developed a plug-in for the algorithm visualization tool AlViE (developed by my supervisor) to allow the user to export the visualization of a particular algorithm in .html format and thus be able to view it in a common browser. The master's thesis (titled ``The Indico KT Project: Improving the Worldwide Impact of Indico'') revolved instead around the Technical Studentship programme I followed at CERN, in 2014: there I was assigned to a series of projects for the Indico application, a web application developed at CERN for events organization. All these projects were part of a more generic project, funded by the Knowledge Transfer group, that was aimed at increasing the visibility of Indico worldwide and its usability. The main projects I pursued during my 14 months at CERN were the Indico Cloud Deployment (i.e. automatizing the process of deploying Indico on a cloud or virtual environment), the Indico Instance Tracking (i.e. the development of a tracking software to track and analyse how the various instances of Indico around the world are used) and the Conference Customization Prototype (to create a prototype for a new conference customization tool that is more intuitive and easy to use). My master's thesis was supervised by the Indico Project Manager Pedro Ferreira (my second referee for this application) at CERN and by Full Professor Pierluigi Crescenzi at the University of Florence. \par \bigskip
    
    During these last years I also developed several other projects. The very first side project I followed at the university was the modelisation of a neural network to mimic the behaviour of a species of intertidal snail (Cerithidea Decollata) that is able to foresee the time and height of incoming tides, in order to try to understand a bit better how this prediction works. When I was at CERN I followed, with a group of students like me, a project, codename Blindstore, that aimed at implementing a protocol to allow private queries to a database (similar to Tor, with the exception that Tor hides who made the query, while Blindstore would hide the query itself). Other minor projects involved the implementation of the learning-to-rank algorithm Rankboost, later included in the Quickrank tool developed at HPC Lab (ISTI, CNR, Pisa), and the implementation of an algorithm to compute the steady-state probabilities of Markov Renewal Processes, later included in the Oris tool developed at STLab (DINFO, University of Florence). \par \bigskip
    
    Recently I took an interest in the areas of system modelling and analysis, especially quantitative analysis. I like a lot the idea of model-based analysis when it comes to predicting various future behaviours of the model and, even more interestingly, scheduling processes on the model in order to achieve various goals. An example could be the study of a maintenance protocol that computes the optimal times for a certain system to be maintained in order to maximize its availability. More generally, I think that as of now there is a vast opportunity for applications of model-based analysis and stochastic processes theory, both in industrial and in research frameworks. In this regard, I have recently participated in the writing of an article that studies how the access latency in Fischer's mutual exclusion protocol varies as some of its parameters are modified, using a model-based analysis built on top of the implementation of the steady-state algorithm for MRPs I worked on. This article, as of today, has been submitted and accepted as short paper and will be presented at MASCOTS 2016, Imperial College, London. On top of that, I've started working right after I got my master's degree (since the 1st of June 2016) as a Research Fellow for STLab (DINFO, University of Florence) working on quantitative analysis of systems. \par \bigskip
    
    As I can see from the available courses for the CSSE curriculum in 2016/17, there are many courses centred on the topics of system modelling and stochastic processes that I can choose. On top of that, I understand that the IMT Institute is an excellence school as well as an advanced international working space where a young student interested in research like me can have a real opportunity to grow both as a researcher in the field of Computer Science as well as a human being. These are the main reasons why I chose to apply for this PhD programme at the IMT Institute. More generally, I look forward pursuing a PhD career as I'm very interested in learning how to do scientific research while doing research in an environment where I can be both autonomous and supervised by experts in the field through my course. \par \bigskip
    
    Regarding the various research fields available for the CSSE curriculum, I'm especially interested in SYSMA, as I believe it's the research field that is the closest to what I'm researching and studying right now. But as you can see I also applied for DYSCO, NETWORKS and PRIAN, as I also like those research fields and would be happy to work on those topics as well. \par \bigskip
    
    I'm certain that the PhD programme offered by IMT, and in particular the CSSE curriculum, are very well suited for me and I strongly believe that I have both the skills and the motivation to pursue this new challenge and get the most out of it. \par \bigskip
    
    It would be an honour to discuss with you my application either personally in Lucca or through a Skype interview. I look forward to hearing from you. \par \bigskip
    
    Thank you in advance for your time and your consideration. \par \bigskip
    
    Sincerely, \par \medskip
    
    \qquad \qquad Tommaso Papini
\end{document}