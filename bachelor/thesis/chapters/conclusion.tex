\chapter*{Conclusioni}
	\addcontentsline{toc}{chapter}{Conclusioni}
	Concludiamo questo studio con alcune considerazioni generali su quanto trattato.\\
	\\
	La quarta versione di AlViE, nata sulla base delle nuove funzioni implementate nel corso di questo lavoro, risulta essere molto in avanti rispetto alla versione precedente: il supporto al Web permette la condivisione su larga scala delle visualizzazioni generate, facilmente inseribili in un qualsiasi sito Web. All'interno del libro di testo Strutture di Dati e Algoritmi (seconda edizione; Pierluigi Crescenzi, Giorgio Gambosi, Roberto Grossi, Gianluca Rossi; Pearson) sono gi� state utilizzate, a scopo didattico, diverse visualizzazioni prodotte con AlViE. Inoltre, sul sito Web di suddetto libro (http://wps.pearsoned.it/crescenzi\_strutture-dati-algoritmi2/220/56566/14480998.cw/index.html) si possono trovare alcune visualizzazioni di AlViE esportate come pagina HTML, ai fini di supportare lo studio e la comprensione delle principali strutture ed algoritmi.\\
	L'editor di strutture dati, invece, aggiunge un nuovo livello d'interazione utente-sistema, facilitando la specifica degli input. Con questa nuova versione vengono anche alleggerite le conoscenze richieste all'utente che voglia definire la propria visualizzazione: adesso, infatti, non � pi� richiesto in alcun modo che l'utente capisca e gestisca la specifica XML delle strutture dati, che saranno completamente occultate agli occhi di quest'ultimo.\\
	\\
	AlViE si sta quindi avviando verso una maggior accessibilit� e semplicit� d'utilizzo, dimostrando di avere buone probabilit� di inserirsi nel contesto mondiale dei sistemi di visualizzazione di algoritmi, tenendo testa ai maggiori esponenti del settore, come JHAV� o OpenDSA.\\
	\\
	Concludendo, il lavoro svolto, dalla costruzione del compilatore e dell'editor di testo, allo studio dell'universo dei sistemi di visualizzazione di algoritmi, si � dimostrato essere molto interessante, a livello sia pedagogico che personale. Questo lavoro � riuscito, infatti, a conciliare pratica e teoria, nel campo della visualizzazione di algoritmi, e, pi� in  generale, dell'Informatica.