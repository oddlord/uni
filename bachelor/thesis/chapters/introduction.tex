\chapter*{Introduzione}
	\pagenumbering{arabic}
	\addcontentsline{toc}{chapter}{Introduzione}
	Gli ultimi 20 anni sono stati caratterizzati da un utilizzo sempre pi� frequente delle tecniche di visualizzazione di algoritmi, da parte dei docenti e di Informatica, dal momento che esse costituiscono un importante strumento per l'apprendimento e la comprensione dei pi� svariati tipi di algoritmo.\\
	Un altro strumento altrettanto importante, per i fini sopra indicati, � rappresentato dal Web. Sono infatti moltissime le applicazioni che permettono, a studenti e ricercatori di tutto il mondo, di scambiarsi e condividere dati e informazioni tramite Internet, facilitandone cos� l'apprendimento e la comprensione, affidandosi anche a strumenti multimediali come la produzione di un'immagine o di un video.\\
	\\
	Il progetto \textit{AlViE4} (\textit{Algorithm Visualization Environment}, ovvero \textit{Ambiente per la Visualizzazione di Algoritmi}) nasce appunto con l'intento di far incontrare queste due realt�, estendendo il programma \textit{AlViE}, per la visualizzazione di algoritmi, al mondo del Web, tramite le nuove possibilit� offerte dallo standard \textit{HTML5}.\\
	\\
	Questo lavoro si compone di quattro capitoli. Nel Capitolo \ref{chap:visualizzazioneAlgoritmi}, per cominciare, effettueremo una panoramica sulle tecniche di visualizzazione esistenti, soffermandosi in particolare sulla loro storia ed evoluzione. Inoltre, verr� proposta una tassonomia per la classificazione ed una maggior comprensione dei sistemi di visualizzazione di algoritmi.\\
	Parleremo poi, nel secondo Capitolo, dello standard \textit{HTML5} (\textit{HyperText Markup Language}, ovvero \textit{Linguaggio di Marcatura degli Ipertesti}): le novit� introdotte rispetto alla versione precedente, le caratteristiche principali e le motivazioni che ci hanno spinto a scegliere gli strumenti messi a disposizione da questo nuovo standard per l'estensione di \textit{AlViE} al mondo del Web.\\
	All'interno del Capitolo \ref{chap:alvie} analizzeremo nello specifico il funzionamento del sistema AlViE, indicandone le novit� introdotte nella quarta versione. Verr� fornito, in conclusione del Capitolo, un esempio passo-passo su come costruire un algoritmo interpretabile da AlViE, producendone quindi una visualizzazione.\\
	In conclusione, vedremo come � stato svolto il lavoro di estensione di AlViE al Web, indicando le tecniche di programmazione utilizzate ed i problemi riscontrati in fase di sviluppo.